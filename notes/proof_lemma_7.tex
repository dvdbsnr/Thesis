%
% Note-Version of Preamble, does not contain chapter reference in theoremstyle definition
%

%%%%% Document class. %%%%
\documentclass [11 pt,	a4paper, notitlepage]{report}

%%%% Packages used. %%%%
% Math packages.
\usepackage{amsfonts}
\usepackage{amsmath} 
\usepackage{amssymb}
\usepackage{amsthm}
\usepackage{mathtools}
\usepackage{bbm}

% Language packages.
\usepackage[english, german]{babel}
\usepackage[ansinew]{inputenc}

% Other packages.
\usepackage{lipsum}
\usepackage{xcolor}
%\usepackage{showkeys}

% Subfig, for figures.
\usepackage{subfig}

% PGFPlot and TikZ.
\usepackage{tikz}
\usepackage{pgfplots}
\usetikzlibrary{positioning,intersections,calc}

% Definition of node styles.
\tikzset{
	plain/.style={draw,circle, fill=white},
	explored/.style = {circle, fill=gray, draw=black, very thick},
	discovered/.style = {circle, draw=black},
	neutral/.style = {circle, draw=black, minimum width=0.5em, minimum height=0.5em}
}


% Select style of todonotes.
\usepackage[disable]{todonotes} % notes not showed
%\usepackage[draft]{todonotes}   % notes showed



%%%% BibLaTeX %%%%
% Set citestyle
%\usepackage[citestyle=authoryear]{biblatex}



%%%% Cosmetics. %%%%
% Set the style of chapter-headers: No "Chapter", only the number.
\usepackage{titlesec}
\titleformat{\chapter}
{\normalfont\LARGE\bfseries}{\thechapter}{1em}{}
\titlespacing*{\chapter}{0pt}{3.5ex plus 1ex minus .2ex}{2.3ex plus .2ex}



%%%% Theorems, Lemmas etc. %%%%
% All Theorems numbered by chapter.
% Theorem.
\newtheorem{theorem}{Theorem}
% Lemma.
\newtheorem{lemma}{Lemma}

% In Definition-style, i.e. no italics.
\theoremstyle{definition}
% Definition.
\newtheorem{definition}{Definition}
%Note, not numbered.
\newtheorem*{note}{Note}



%%%% Commands %%%%
% BigO notation
\newcommand{\BigO}[1]
	{O\left(#1\right)}
\newcommand{\Smallo}[1]
	{o\left(#1\right)}
\newcommand{\OrderO}[1]
	{\Theta \left(#1\right)}

% n{1,3} -> n^{1/3}]
\newcommand{\n}[2]
	{n^{#1 / #2}}
	
% Zn -> Z_n()
\newcommand{\Zn}[1]
	{Z_n(#1)}
% An -> A_n()
\newcommand{\An}[1]
	{A_n(#1)}
% Bn -> B_n()
\newcommand{\Bn}[1]
	{B_n(#1)}
% Mn -> M_n()
\newcommand{\Mn}[1]
	{M_n(#1)}
% Quadratic variation of Mn: [Mn]t
\newcommand{\Mnq}[1]
	{[M_n]_{#1}}
% an -> a_n()
\newcommand{\an}[1]
	{a_n(#1)}
% amn -> a'_n() (a mark n)
\newcommand{\amn}[1]
	{a'_n(#1)}
% Ft filtration
\newcommand{\F}[1]
	{\mathcal{F}_{#1}}
% Ft filtration
\newcommand{\Fn}[1]
	{\mathcal{F}^n_{#1}}

% Z, M, A, Q, B bar
\newcommand{\Zbar}[1]
    {\bar{Z}_n(#1)}
\newcommand{\Abar}[1]
    {\bar{A}_n(#1)}
\newcommand{\Mbar}[1]
    {\bar{M}_n(#1)}
\newcommand{\Bbar}[1]
    {\bar{B}_n(#1)}
    
% Deterministic lemma
\newcommand{\tn}[1]
	{t_{n, #1}}
\newcommand{\tip}
	{t_{i+1}}
\newcommand{\Xin}
	{\Xi^{(n)}}
\newcommand{\xip}{x_{i+1}}


% Zeta = C^-1_n()
\newcommand{\Zeta}[1]
	{\zeta^{-1}_n(#1)}
	
% E_s,ds, event of new edge
\newcommand{\Esds}
	{\mathcal{E}_{[s,s+ds]}	}
% \ps = p(n)/(1-(s-s)p(n))
\newcommand{\ps}
	{\frac{p(n)}{1-(s-\floor{s})p(n)}}
% \psx{x} = xp(n)/(1-(s-s)p(n))
\newcommand{\psx}[1]
	{\frac{#1 p(n)}{1-(s-\floor{s})p(n)}}
	
% \supns sup_{s<=n^2/3s0}
\newcommand{\supns}
	{\sup_{s \leq \n{2}{3} s_0 } }
	
% Condition in Expected Value, adds line and spaces.
\newcommand{\cond}
	{\: | \:}
% Expected value, regular brackets
\newcommand{\Exp}[1]
	{\mathbb{E}[#1]}
% Expected value, big fitted brackets
\newcommand{\ExpBig}[1]
	{\mathbb{E} \left[ #1 \right]}
% Ceiling
\DeclarePairedDelimiter\ceil{\lceil}{\rceil}
% Floor
\DeclarePairedDelimiter\floor{\lfloor}{\rfloor}

% Real numbers
\newcommand{\Real}
    {\mathbb{R}}
% Natural numbers
\newcommand{\Nat}
    {\mathbb{N}}



\begin{document}
\begin{lemma}
Let $f:[0, \infty) \rightarrow \Real$ be a continuous function. 
Let $\mathcal{E}$ be the set of non-empty intervals 
$e=(l,r) \subset \Real_{\geq 0}$
such that
\begin{equation} \label{E: cond 1}
f(r) = f(l) = \min_{s \leq l} f(s),
\end{equation}
\begin{equation} \label{E: cond 2}
f(l) < f(s) \quad \forall l < s < r.
\end{equation}
Define $\Xi := \lbrace (l, r-l) \; | \; (l, r) \in \mathcal{E} \rbrace$.

Suppose that for intervals $e_1, e_2 \in \mathcal{E}$ we have 
\begin{equation} \label{E: f(l1) > f(l2)}
f(l_1) > f(l_2)
\end{equation}
and the complement of $\cup_{e \in \mathcal{E}} (l,r)$ has Lebesgue measure zero,
\begin{equation} \label{E: complement zero}
\mu \left( \left( \cup_{e \in \mathcal{E}} (l,r) \right)^c\right) = 0.
\end{equation}

Let $f_n \rightarrow f$ as $n \rightarrow \infty$ uniformly on bounded intervals.
Suppose $(\tn{i}, i \geq 1)$ satisfy the following conditions:
\begin{align}
&0 = \tn{1} < \tn{2} < ... \; \text{and} \; \lim_{i \rightarrow \infty} \tn{i} = \infty, \label{E: cond 3} \\ 
&f_n(\tn{i}) = \min_{u \leq \tn{i}}f_n(u), \label{E: cond 4} \\ 
&\max_{i: \tn{i} \leq s_0}(f_n(\tn{i}) - f_n(\tn{i+1})) \rightarrow 0 \; \text{as} \; n \rightarrow \infty, \; \text{for all} \; s_0 \leq \infty. \label{E: cond 5}
\end{align}
Define $\Xin := \lbrace (\tn{i}, \tn{i+1} - \tn{i}) \; | \; i \geq 1 \rbrace$.
Then $\Xin \rightarrow \Xi$ as $n \rightarrow \infty$.
\end{lemma}


\begin{proof}
We need to proof that
\begin{equation}
\begin{aligned}
\lim_{n \rightarrow \infty} \Xin 
&:= \lbrace x \in \Real^2 \; | \; \lim_{n \rightarrow \infty} \mathbbm{1}_{\Xin}(x) = 1 \rbrace \\ 
&= \Xi.
\end{aligned}
\end{equation}
\begin{description} %TWO PARTS
% FIRST PART
\item[$\lim_{n \rightarrow \infty} \Xin \supseteq \Xi$:] 
Let $(l, r-l) \in \Xi$ and thus $(l, r) \in \mathcal{E}$.
We show that $(l,r)$ satisfy conditions \eqref{E: cond 3} to \eqref{E: cond 5} for all $n$ greater than some $N_0 \in \Nat$.

\begin{description} % CONDITIONS 1-3
	\item[\eqref{E: cond 3}:] 
Since the interval is non-empty, $l < r$ holds
and, by conditions \eqref{E: complement zero},
for all $K \in \Real$ there is $(l,r) \in \mathcal{E}$ such that $r > K$.
The interval $(0,r)$ is an element of $\mathcal{E}$ for some $r>0$, 
since $f(0) = min_{u \leq 0}f(u)$
and otherwise there would be an $l>0$ such that $f$ is decreasing monotonically on $[0,l]$,
which is a contradiction to \eqref{E: complement zero}.
	\item[\eqref{E: cond 4}:]
Let $x < l$. Thus $f(x) \geq f(l)$. 
Suppose $\forall N_0 \; \exists n \geq N_0$ such that
$f_n(x) < f_n(l)$.
Then $f_n(x) \nrightarrow f(x)$, 
which is a contradiction to the uniform convergence.
Thus $\exists N_0$ such that $\forall n \geq N_0$: 
$f_n(x) \geq f_n(l)$, so
$f_n(l) = \min_{u \leq l}f_n(u)$.
	\item[\eqref{E: cond 5}:]
Define $M:=\lbrace x \in \Real \; | \; x=l \; \text{or} \; x=r \; \text{for some} \; (l,r) \in \mathcal{E} \rbrace$.
We will show that $\exists N_0$ such that $\forall n \geq N_0$:
\begin{equation*}
\max_{\substack{x_i, \xip \in M, \\ x_i, \xip \leq s_0}} (f_n(x_i) - f_n(\xip)) \rightarrow 0
\end{equation*}
where $\xip$ is the smallest element in $M$ greater than $x_i$.
Since $f_n \rightarrow f$ uniformly on bounded intervals, 
$|f_n(x) - f(x)| < \epsilon$ for all $x \leq s_0$ and $n \geq N_0$.
So for $x_i, \xip \leq s_0$:
\begin{equation}
\begin{aligned}
f_n(x_i) - f_n(\xip) 
&\leq f(x_i) + |f_n(x_i) - f(x_i)| \\
&\quad - f(\xip) + |f_n(\xip) -f(\xip)| \\
&\leq f(x_i) - f(\xip) + 2\epsilon.
\end{aligned} 
\end{equation}
Now $f(x_i) - f(\xip) = 0$ for both $x_i = l$, $\xip = r$ 
and $x_i = r_1$, $\xip = l_2$. 
The former since $f(l) = f(r)$ for $(l,r) \in \mathcal{E}$,
the latter since $\mu \left(\cup_{e \in \mathcal{E}} (l,r)\right) = 0$
thus $\mu \left( (r_1, l_2) \right) = 0$.

So $f_n(x_i) - f_n(\xip) \rightarrow 0$ for all $x_i, \xip$ and thus 
\begin{equation*}
\max_{\substack{x_i, \xip \in M, \\ x_i, \xip \leq s_0}} (f_n(x_i) - f_n(\xip)) \rightarrow 0.
\end{equation*}
\end{description} % CONDITIONS 1-3
Now $l$ and $r$ are elements of $(\tn{i}, i \geq 1) =: S_n$ for all sufficiently large $n$.
We show that there can be no other element between them, 
so $(l, r-l) = (\tn{i}, \tn{i+1} - \tn{i}) \in \Xin$.
Suppose there exists $t \in S_n$ such that $l < t < r$.
For \eqref{E: cond 2} $f(t) > f(l) = f(r)$, so
\begin{equation*}
f_n(t) > f_n(l) >= \min_{u \leq l} f_n(u) >= \min_{u \leq t} f_n(u)
\end{equation*}
for sufficiently large $n$. So $t$ can not be an element of $S_n$.

% SECOND PART
\item[$\lim_{n \rightarrow \infty} \Xin \subseteq \Xi$:]
Let $(t_i, \tip - t_i) \in \Xin$ and $(t_i, \tip) \in S_n$
for all $n$ greater than some $N_0$.
We will show that
\begin{equation}
(t_i, \tip ) = (l,r),
\end{equation}
for some $(l,r) \in \mathcal{E}$,
by proving that $(t_i, \tip)$ satisfies conditions \eqref{E: cond 1} and \eqref{E: cond 2}.


\begin{description} % CONDITIONS 1 & 2
	\item[\eqref{E: cond 1}:]
\begin{equation}
\begin{aligned}
f(t_i) - f(\tip) 
&\leq f_n(t_i) + |f(t_i) - f_n(t_i)| \\
&\quad- f_n(\tip) + |f(\tip) - f_n(\tip)| \\
&\leq f_n(t_i) - f_n(\tip) + 2\epsilon \\
&\leq 3\epsilon,
\end{aligned}
\end{equation}
by the uniform convergence $f_n \rightarrow f$ and condition \eqref{E: cond 5}.
So
\begin{equation}
f(t_i) = f(\tip).
\end{equation}

	\item[\eqref{E: cond 2}:]
Suppose there exists $t_i < s < \tip$ such that
$f(s) < f(t_i)$
then
\begin{equation*}
f_n(\tip) = f_n(t_i) > f_n(s) \geq \min_{u \leq \tip}f_n(u),
\end{equation*}
for sufficiently large $n$, which is a contradiction to \eqref{E: cond 4}.

Suppose $f(s) = f(t_i)$. Then $s \in S_n$ for all $n \geq N_0$.
But since $t_i < s < \tip$ now $(t_i, \tip - t_i) \notin \Xin$,
since only $(t_i, s - t_i)$ and $(s, \tip - s)$ are elements of $\Xin$.
\end{description} % CONDITIONS 1 & 2
\end{description} % TWO PARTS
\end{proof}
\end{document}