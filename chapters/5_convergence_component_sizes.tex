% Chapter 4: Convergence of component sizes
% Contains:
%   Explanation on what's still missing
%   The deterministic lemma (Lemma 7)
%   Girsanov Theorem
%   Applying Lemma 7 (Lemma 8)
%   Graph theory approach:
%     Lemma 9
%   l2 approach:
%     Definitions on l2
%     Deterministic lemma (Lemma 14)
%     Proposition 15
%     Applying Prop 15 to the bf-walk

\chapter{Convergence of component sizes and surplus edges} \label{C: convergence}
\fxnote{Update title.}

%%%%%%%%%%%%%%%%%%%%%%%%%%%%%%%%%%%%%%%%%%%%%%%%%%%%%%%%%%%%
% What's still missing
%%%%%%%%%%%%%%%%%%%%%%%%%%%%%%%%%%%%%%%%%%%%%%%%%%%%%%%%%%%%

In Chapter~\ref{C: bf-walk} we have shown that the rescaled breadth-first walk $\rznt$ on $\Gnt$
converges in distribution to the Brownian motion with drift $\Bt$.
Intuitively it is clear that Theorem~\ref{T: Main} should follow:
Component sizes are coded into the breadth-first walk as excursions above past minima,
excursions of $\Bt$ are excursions of $\Wt$ above past minima.
But rigorously deducing a proof of our main theorem requires a bit more work.
To make sure that indeed components and excursions do match up,
we describe them as two-dimensional point processes
in which the first entry gives the start of an excursion or component, 
the second the length of an excursion or the size of a component.
The following two Lemmas in Section~\ref{S: matching components} prove that this sequence of Poisson point processes describing components 
converges to the point process describing excursions with regard to the vague topology.

This implies a convergence of all excursions of sufficient length on finite intervals.
For a proof of Theorem~\ref{T: Main} to hold, it remains to be shown that we do not need to worry about "mass wandering off to infinity",
meaning the rescaled starting points of excursions of sufficient length to diverge to infinity as $n \rightarrow \infty$.
Another problem that might arise is the mass of all small excursions prohibiting a convergence in $\ld$.
We deal with both problems in Section~\ref{S: late excursions} and at last prove our main theorem in Section~\ref{S: proofs}.


\section{Matching components} \label{S: matching components}

%%%%%%%%%%%%%%%%%%%%%%%%%%%%%%%%%%%%%%%%%%%%%%%%%%%%%%%%%%%%
% Deterministic Lemma 7: Statement and Note
%%%%%%%%%%%%%%%%%%%%%%%%%%%%%%%%%%%%%%%%%%%%%%%%%%%%%%%%%%%%

We start with a deterministic Lemma.
Given a continuous function $f$ with properties similar to a Brownian motion
and a sequence of functions $f_n$ converging to $f$ uniformly on bounded intervals,
we can define a set of excursions on each $f_n$ such that
the point processes of starts and lengths of excursions converge vaguely to the point processes of starts and lengths of excursions on $f$.

\begin{lemma} \label{L: Deterministic Lemma}
	Let $f:[0, \infty) \rightarrow \Real$ be a continuous function. 
	Let $\mathcal{E}$ be the set of non-empty intervals 
	$e=[l,r] \subset \Real_{\geq 0}$
	such that
	\begin{equation} \label{E: f cond 1}
	f(r) = f(l) = \min_{s \leq l} f(s),
	\end{equation}
	\begin{equation} \label{E: f cond 2}
	f(l) < f(s) \quad \forall l < s < r.
	\end{equation}
	Define $\Xi := \lbrace (l, r-l) \; | \; (l, r) \in \mathcal{E} \rbrace$.
	
	Suppose that for intervals $(l_1, r_1), (l_2, r_2) \in \mathcal{E}$ with $l_1 < l_2$ we have 
	\begin{equation} \label{E: f cond f(l1) > f(l2)}
	f(l_1) > f(l_2)
	\end{equation}
	and the complement of $\cup_{e \in \mathcal{E}} (l,r)$ has Lebesgue measure zero,
	\begin{equation} \label{E: f cond complement zero}
	\mu \left( \left( \cup_{e \in \mathcal{E}} (l,r) \right)^c\right) = 0.
	\end{equation}
	
	Let $f_n \rightarrow f$ as $n \rightarrow \infty$ uniformly on bounded intervals.
	Now define $\Xin := \lbrace (\tn{i}, \tn{i+1} - \tn{i}) \; | \; i \geq 1 \rbrace$
	for any sequence of sets of points $\fnPoints := (\tn{i}, i \geq 1)$ satisfying the following conditions:
	
	\begin{equation} \label{E: fn cond 3}
	0 = \tn{1} < \tn{2} < ... \; \text{and} \; \lim_{i \rightarrow \infty} \tn{i} = \infty,
	\end{equation}
	\begin{equation} \label{E: fn cond 4}
	f_n(\tn{i}) = \min_{u \leq \tn{i}}f_n(u), 
	\end{equation}
	\begin{equation} \label{E: fn cond 5}
	\max_{i: \tn{i} \leq s_0}(f_n(\tn{i}) - f_n(\tn{i+1})) \rightarrow 0 \; \text{as} \; n \rightarrow \infty, \; \text{for all} \; s_0 < \infty.
	\end{equation}
	
	Then $\Xin \rightarrow_v \Xi$ as $n \rightarrow \infty$.
\end{lemma}

\begin{note}
	The convergence
	$\Xin \rightarrow_v \Xi$
	is to be interpreted as the vague convergence of counting measures
	which, by Lemma~\ref{L: portmanteau vague},
	is equivalent to the convergence
	\begin{equation}
	\Xin(C) \rightarrow \Xi(C)
	\end{equation}
	for all relatively compact subsets
	$C \subseteq \IntTimesInt$,
	with $\Xi(\partial C) = 0$.
	In this case, a relatively compact subset is a pair of intervals
	$[T_1, T_2] \times [d_1, d_2]$,
	with $T_1, T_2 \geq 0$
	and $d_1, d_2 > 0$.
	The condition on the measure of the boundary means
	that the limit process must not have any excursions starting exactly at
	$T_1$ or $T_2$,
	or any excursions of length exactly $d_1$ or $d_2$.
	An exception is the case of $T_1=0$.
	Since the domain of $f$ starts at $0$
	and the condition on the boundary is needed to prevent the case of points in $\Xin$ converging to some point on the boundary "from outside",
	we do not need to consider $0$ as part of the boundary of an interval 
	$[0, T_2] \times [d_1, d_2]$.
\end{note}

%%%%%%%%%%%%%%%%%%%%%%%%%%%%%%%%%%%%%%%%%%%%%%%%%%%%%%%%%%%%
% Deterministic Lemma 7: Proof
%%%%%%%%%%%%%%%%%%%%%%%%%%%%%%%%%%%%%%%%%%%%%%%%%%%%%%%%%%%%
\begin{proof}
%	Structure:
%	- What does the definition of \Xi and \Xin mean?
%	- Detail on \Xin and conditions
%   First part: \Xi in \Xin
%   (For every f exc, there is an fn exc)
%	- Facts about f:
%	 - It goes up to the left and right of l
%	 - It goes down to the right of r? (Or later)
%	 - eps/delta stuff
%	- Arch of fn
%	- Definition ln(x)
%	- ln(x) in [l-e, l+e]
%	- Definition rn(x)
%	- rn(x) in [r-e, r+e]
To prove the Lemma, 
we fix some bounded subset of
$\IntTimesInt$,
$C := [T_1, T_2] \times [d_1, d_2]$,
and show that, for sufficiently large $n$,
$\Xin(C) = \Xi(C)$,
that is,
there are exactly as many excursions of $f_n$ starting in
$(T_1, T_2)$, 
with length in $(d_1, d_2)$,
as similar excursions of the limit function $f$.

We will first show that every excursion of $f$ is eventually matched by some excursion of $f_n$,
then show that there can not be any more excursion of $f_n$ of sufficient length.


%%%%%%%%%%%%%%%%%%%%%%%%%%%%%%
% Proof Part 1: \Xi in \Xin
%%%%%%%%%%%%%%%%%%%%%%%%%%%%%%
\begin{proof}[Part 1 ($\Xi(C) \leq \Xin(C)$)] \label{PP: Lemma Det 1} \renewcommand{\qedsymbol}{}
Let us first establish some facts about excursions of the limit function $f$.
In the interval $[T_1, T_2]$,
there can only be a finite number of excursion starting with length of at least $d_1$,
at most $(T_2 - T_1)/d_1$.
We call excursions of length greater than $d_1$ large, all other excursions small.
Let
\begin{equation}
\mathcal{E}^* := \{ (l_i, r_i) \; | \; i=1, \dots, k \}	
\end{equation}

be the set of these excursions.
Consider a
$(l,r) \in \mathcal{E}^*$.
Since $l-r > d_1$,
we can find an $\epsilon > 0$
such that the length of the interval
$[l+\epsilon, r-\epsilon]$
is still greater than $d_1$.

We show that we can find an $\epsilon > 0$ sufficiently small,
so that for every
$x \in [0, l-\epsilon] \cup [l+\epsilon, r-\epsilon]$,
\begin{equation} \label{E: fx > fl + delta}
f(x) > f(l) + \delta
\end{equation}
holds for some $\delta > 0$:

Assume this does not hold to the left of the $\epsilon$-neighbourhood of $l$.
Then for all $\delta > 0$ exists $x_{\delta} \in [0, l - \epsilon]$ such that
$f(x_{\delta}) \leq f(l) + \delta$.
Take $\delta_n = \frac{1}{n}$. 
Then $x_{\delta_n}$ is a sequence in the compact interval $[0, l - \epsilon]$
and the Bolzano–Weierstrass theorem states that there exists a convergent subsequence with limit $x \in [0, l - \epsilon]$.
Then $f(x_{\delta}) \leq f(l) + \delta$ for all $\delta > 0$.
This implies $f(x) \leq f(l)$ and by condition \eqref{E: f cond 1},
$f(l) = \min_{u \leq l}f(u)$,
so $f(x) = f(l)$.
That leaves two possibilities:
First, $f$ is constant on the interval $[x,l]$.
But this would be an interval of non-zero length without any excursions on it,
a contradiction to \eqref{E: f cond complement zero}.
Second, there is an interval
$[x', l']$, with $x\leq x'<l' \leq l$,
such that $f(y) > f(x)$ for all $y \in (x',l')$.
That makes $(x',l')$ another excursion in $\mathcal{E}$,
but $f(x') = f(l)$,
which is a contradiction to condition \eqref{E: f cond f(l1) > f(l2)}.

Using the same logic,
assuming \eqref{E: fx > fl + delta} does not hold in $[l + \epsilon, r- \epsilon]$
leads to a point $x \in [l + \epsilon, r- \epsilon]$
such that $f(x) \leq f(l)$,
which is a contradiction to condition \eqref{E: f cond 2}.

Now consider the behaviour of $f$ on $[r, r + \epsilon]$.
As previously stated, 
condition \eqref{E: f cond complement zero} prevents $f$ from being constant.
If $f(x) \geq f(r)$ for all $x \in [r, r + \epsilon]$,
there would be another excursion that contradicts \eqref{E: f cond f(l1) > f(l2)}.
So there must exist an $r^* \in (r, r + \epsilon]$ with $f(r^*) < f(r)$.

\bigskip

We take a look at $f_n$.
Fix an $x^* \in [l+\epsilon, r-\epsilon]$.
We will now show that there exist points
$\lnx \in [l-\epsilon, l+\epsilon]$
and 
$\rnx \in [r-\epsilon, r+\epsilon]$
for which condition \eqref{E: fn cond 4} holds,
making $(\lnx, \rnx- \lnx)$ a possible element of $\Xin$.

We define 
\begin{equation} \label{D: lnx}
\lnx := \min\{\argmin_{u \leq x^*} f_n(u)\}
\end{equation}
and 
\begin{equation} \label{D: delta star}
\delta^* := \min\{ \delta, f(r) - f(r^*) \},
\end{equation}
where $\delta$ is the constant used in the discussion of
$[0, l-\epsilon]$ and $[l+\epsilon, r-\epsilon]$ above.

By the convergence $f_n \rightarrow f$, 
which is uniform on the compact interval $[T_1, T_2]$, 
we can find an $N \in \Nat$ such that 
\begin{equation} \label{E: fn f delta star/3}
|f_n(x) - f(x)| < \delta^* / 3
\end{equation} for all
$x \in [T_1, T_2]$ and $n \geq N$.
So for every point $x \in [0, l - \epsilon]\cup[l + \epsilon, r - \epsilon]$,
\begin{equation} \label{E: fn(x) > fn(l)}
\begin{aligned}
f_n(x) 
&> f(x) - \delta^* / 3  &\text{by \eqref{E: fn f delta star/3}\;}\\
&> f(l) + \delta^* - \delta^* / 3  &\text{by \eqref{D: delta star}\;}\\
&> f(l) + \delta^* / 3 &\\
&> f_n(l)  &\text{by \eqref{E: fn f delta star/3}}.
\end{aligned}
\end{equation}
So $f_n$ takes its minimum over 
$[0, l - \epsilon]\cup[l + \epsilon, r - \epsilon]$
at $l$ or somewhere in $[l-\epsilon, l+\epsilon]$.
\fxnote{Maybe give the interval a name.}
Therefore $\lnx \in [l-\epsilon, l+\epsilon]$ and $f_n(\lnx) = \min_{u \leq \lnx} f_n(u)$.

\bigskip

For the right side of the interval, define
\begin{equation} \label{D: rnx}
\rnx := \inf \{ x > x^* \; | \; f_n(x) = \lnx \}.
\end{equation}
By \eqref{E: fn(x) > fn(l)}, the function $f_n$ can not reach its past minimum $f_n(\lnx)$
before $r-\epsilon$.
As previously discussed,
there exists a $r^* \in (r, r+\epsilon]$ such that $f(r) > f(r^*)$ 
and by \eqref{D: delta star} we have $f(r) - f(r^*) \geq \delta^*$.
We calculate
\begin{equation}
\begin{aligned}
f_n(\lnx) 
&> f(l) - \delta^* / 3 \quad &\text{by \eqref{E: fn f delta star/3}\;}\\
&= f(r) - \delta^* / 3 \quad &\text{by \eqref{E: f cond 1}\;\;\;}\\
&\geq f(r^*) + \delta^* - \delta^* / 3 \quad &\text{by \eqref{D: delta star}\;} \\
&> f(r^*) + \delta^*/3 \\
&> f_n(r^*) \quad &\text{by \eqref{E: fn f delta star/3}},
\end{aligned}
\end{equation}
\fxnote{Align these.}
which implies that $\rnx$ must be smaller than $r^*$,
since $f_n$ is a continuous function and must cross $f_n(\lnx)$ before becoming smaller at $r^*$.
Therefore $\rnx \in [r - \epsilon, r+ \epsilon]$.
Since $f_n(l_n(x)) = \min_{u \leq \lnx} f_n(u)$
and $\rnx$ is the first time this previous minimum is reached,
$f_n(\rnx) = \min_{u \leq \lnx} f_n(u)$.

\bigskip

We have shown that the only points satisfying condition \eqref{E: fn cond 4} must lay near the beginning and end of excursions of $f$.
This is not sufficient as proof of the Lemma, since the sequence $\fnPoints$ might not contain any points between two large excursions,
or even not contain any points, thus skipping one or more eligible excursions.
We will now show that \eqref{E: fn cond 3} and \eqref{E: fn cond 5} imply 
that any set satisfying these two conditions must contain at least one element in between two large excursions of $f$.

First of all, \eqref{E: fn cond 3} ensures there must exist points in $\fnPoints$ and no last element of $\fnPoints$ can exist.

Consider two consecutive large excursions of $f$, $(l_1, r_1)$ and $(l_2, r_2)$
and the space between $r_1$ and $l_2$.
Suppose there is no element of $\fnPoints$ in $[r_1-\epsilon, l_2+\epsilon]$ for all $\epsilon>0$.
The latest element of $\fnPoints$ was located at or before $l_1$, the next will be at or around $r_2$.
We know $f(r_1)=f(l_1) > f(l_2)$, so there is $\delta>0$ such that for all large $n$,
$|f_n(l^*) - f_n(r^*)|> \delta$ for all $l^*$ and $r^*$ in the sufficiently small $\epsilon$-neighbourhoods around $l_1$ and $r_2$.

Any previous element of $\fnPoints$ will only yield a larger, any element after $r_2$ only a smaller $f_n$-value.
This is a contradiction to \eqref{E: fn cond 5}.
So there must be at least one point of $\fnPoints$ in between these two large excursions, no excursion can be skipped.

\bigskip

For any excursion $(l,r)$ of $f$, there exists $\lnx, \rnx \in \fnPoints$ in the respective $\epsilon$-neighbourhoods,
so the excursion of $f$ is matched by an excursion of $f_n$ of similar length.
Since $l - r + 2\epsilon > d_1$,
we have found $\tn{i} = \lnx$, $\tn{i+1} = \rnx$,
such that $(\tn{i}, \tn{i+1} - \tn{i}) \in \Xin \cap C$
for all $n$ greater than some $N_i = N \in \Nat$.
Now let $N^* := \max\{N_1, \dots, N_k\}$,
where $k$ is the finite number of elements of $\mathcal{E}^*$,
and every excursion of $f$ in $C$ is matched by an excursion of $f_n$ in $C$
for $n \geq N^*$.
Therefore eventually $\Xi (C) \leq \Xin(C)$.
\end{proof}


%%%%%%%%%%%%%%%%%%%%%%%%%%%%%%
% Proof Part 2: \Xin in \Xi
%%%%%%%%%%%%%%%%%%%%%%%%%%%%%%
\begin{proof}[Part 2 ($\Xin(C) \leq \Xi(C)$)] \label{PP: Xin in Xi} \label{PP: Lemma Det 2} \renewcommand{\qedsymbol}{}
Now that every excursion of $f$ is matched, 
we need to show that there can not exist any additional large excursion of $f_n$.
Considering the fact that excursions can not overlap,
the only possibility for an additional large excursion is the space between two large excursions.
For a pair of large excursions,
$(l_i, r_i)$ and $(l_{i+1}, r_{i+1})$,
there is only enough space in between them if
$r_i - l_{i+1} > d_1$.
Consider such an interval $[r,l]$ of length greater than $d_1$.
By condition \eqref{E: f cond complement zero} the space must be filled with smaller excursions of $f$.
There is an at most countable number of these,
\fxnote{Why?}
let $\mathcal{E^*} := \{ (l_i, r_i) \; | \; i \in \Nat \}$
be the set of such excursions starting in $(r,l)$ with length less than or equal to $d_1$.
We know
\begin{equation} 
\sum_{i=1}^{\infty} r_i - l_i = l - r > d_1,
\end{equation}
so we can choose a finite set of excursions
$\{ (l_i, r_i) \; | \; i = 1, \dots, K \}$
such that, 
if we exclude these from the interval $[r,l]$, 
the space remaining is less than $d_1$:
\begin{equation}
l-r - \sum_{i=1}^{K} r_i - l_i < d_1.
\end{equation}
Let $d^* < \min \{ r_i - l_i \; | \; i = 1, \dots, K \}$
and apply the logic of Part~\ref{PP: Lemma Det 1} to the compact set
$[r, l] \times [d^*, d_1]$.
For sufficiently large $n$, 
every one of these $K$ excursions of $f$ will be matched with an excursion of $f_n$,
so that there will be no space left for a large excursion of $f_n$ in $[r,l]$.
Applying this logic to every one if the finitely many gaps between excursions of $f$,
we see that there can not exist any more large excursions of $f_n$  than those already matching $f$.
There $\Xin(C) \leq \Xi(C)$ for sufficiently large $n$, which completes the proof.
\end{proof}
\end{proof}



%%%%%%%%%%%%%%%%%%%%%%%%%%%%%%%%%%%%%%%%%%%%%%%%%%%%%%%%%%%%
% Lemma 8: Statement
%%%%%%%%%%%%%%%%%%%%%%%%%%%%%%%%%%%%%%%%%%%%%%%%%%%%%%%%%%%%
The following lemma will now link excursions of $\rznt$ and $\Wt$ in the language of Lemma~\ref{L: Deterministic Lemma}.
We define a random point process containing the starts and lengths of excursions of $\Bt$ together with a sequence of random point processes describing the starts and sizes of components discovered by $\rznt$.
Let $\Ci{n,i}$ be the size of the $i$-th component and
define $\gamma_n(i) \in \{1, \dots, n\}$ as the index for which
$v_{\gamma_n(i)}$ is the last vertex of the $i-1$-th component encountered by the breadth-first walk $Z_n$.
For this lemma, we need to extend the definitions of $\gamma_n$ and $\Ci{n,i}$ beyond the $n$-th vertex and the last component.
Let $i^*_n$ be the index of the last component 
and define $\gamma_n(i^*_n + 1) := \max_{1 \leq i \leq i^*_n}\gamma_n(i) + 1$ 
and consequently $\gamma_n(j) := \gamma_n(j-1) + 1$ for all $j > i^*_n + 1$.
For component sizes, let $\Ci{n,i} := 1$ for all $i > i^*$.


\begin{lemma} \label{L: Lemma 8}
	Let $\Xi$ be the point process with points corresponding to excursions of $\Bt$,
	\begin{equation} \label{E: Lemma 8 def Xi}
	\Xi := \{ (l(\gamma), |\gamma|) \; | \; \gamma \; \text{excursion of} \; \Bt \}.
	\end{equation}
	Let $\Xin$ be the rescaled point process with points corresponding to excursions of the breadth-first walk
	\begin{equation} \label{E: Lemma 8 def Xin}
	\Xin := \{ ( \n{-2}{3} \gamma_n(i), \n{-2}{3} \Ci{n,i} ) \; | \; i \geq 1 \}.
	\end{equation}
	Then $\Xin \rightarrow_d \Xi$ as $n \rightarrow \infty$ with regard to the vague topology.
\end{lemma}

%%%%%%%%%%%%%%%%%%%%%%%%%%%%%%%%%%%%%%%%%%%%%%%%%%%%%%%%%%%%
% Lemma 8: Proof
%%%%%%%%%%%%%%%%%%%%%%%%%%%%%%%%%%%%%%%%%%%%%%%%%%%%%%%%%%%%
\begin{proof}
	First, let us give some remarks on the structure of the objects in use here and their convergence.
	Lemma~\ref{L: Deterministic Lemma} states that, under certain conditions, two deterministic point processes converge in the vague topology.
	This Lemma now introduces $\Xin$ and $\Xi$, which are random variables mapping into the space of point processes, again equipped with the vague topology.
	
	Both $\Xin$ and $\Xi$ are random in the sense that they depend on their underlying processes, $\rznt$ and $\Wt$ respectively,
	so to clarify we define the sets
	\begin{equation} \label{E: Lemma 8 def XiWt}
	\begin{aligned}
	\Xi_{\Wt} &:= \{ (l(\gamma), |\gamma|) \; | \; \gamma \; \text{excursion of} \; \Bt\}, \\
	\Xin_{\znt} &:= \{ ( \n{-2}{3} \gamma_n(i), \n{-2}{3} \Ci{n,i} ) \; | \; i \geq 1 \}, \\
	\bar{\Xi}^{(n)}_{\rznt} &:= \{ ( \bar{\gamma}_n(i), \bar{\Ccal}_{n,i} ) \; | \; i \geq 1 \},
	\end{aligned}
	\end{equation}
	which are dependent on the specific realisation of the random processes $\Wt$ and $\znt$,
	where $\bar{\gamma}_n$ and $\bar{\Ccal}_{n,i}$ describe starting points and excursion lengths of $\rznt$.
	Since $\rznt(s) = \n{-1}{3}\znt(\n{2}{3}s)$ we have $\Xin_{\znt} = \bar{\Xi}^{(n)}_{\rznt}$
	and we can use $\Xin$ and $\bar{\Xi}^{(n)}$ interchangeably when appropriate.
	
	Now \eqref{E: Lemma 8 def Xi} and \eqref{E: Lemma 8 def Xin} can be redefined as the random variables
	\begin{equation} \label{E: Lemma 8 def Xi Xin}
	\begin{aligned}
	\Xi : \; &\omega \mapsto \Xi_{\Wt(\omega)}, \\
	\Xin : \; &\omega \mapsto \Xin_{\znt(\omega)}.
	\end{aligned}
	\end{equation}
	
	We apply the reasoning used in the proof of Theorem~\ref{T: Joint Convergence} once more,
	and use the Skorohod representation theorem to construct random variables
	$\rzntX$ and $\WtX$ on a probability space $(\Omega', \mathcal{F}', \mathbb{P}')$,
	which converge almost surely.
	Almost sure convergence implies
	\begin{equation}
	\sup_{s \leq s_0}|\rzntX(s) - \WtX(s)| \rightarrow 0
	\end{equation}
	for all $s_0 < \infty$ almost surely,
	so $\rzntX(\omega') \rightarrow \WtX(\omega')$ uniformly on bounded intervals for almost all $\omega' \in \Omega'$. 
	Analogously we define
	\begin{equation}
		\BtX(s) := \WtX(s) - \min_{u \leq s}\WtX(s).
	\end{equation}
	
	On the same probability space, we now define $\XinX$ and $\XiX$ as in \eqref{E: Lemma 8 def XiWt} and \eqref{E: Lemma 8 def Xi Xin}.
	Since $\rznt \sim \rzntX$ and $\Wt \sim \WtX$ we have $\Xin \sim \XinX$ and $\Xi \sim \XiX$.
	
	By the definition of $\BtX$, 
	$\XiX$ is the $\Xi$ in Lemma~\ref{L: Deterministic Lemma} with $f = \WtX$.
	As $f_n$ we take $\rznt$ and define $\tn{i} := \n{-2}{3} \gamma(n,i)$, that is, 
	the elements of $\fnPoints$ to be the end-points of components,
	rescaled to match $\rzntX$.
	This way $\XinX$ coincides with $\Xin$ in Lemma~\ref{L: Deterministic Lemma}.
	
	We still need to show that conditions \eqref{E: f cond 1} to \eqref{E: f cond complement zero} hold for $\WtX$
	and conditions \eqref{E: fn cond 3} to \eqref{E: fn cond 5} hold for the breadth-first walk.
	
	%%%%%%%%%%%%%%%%%%%%%%%%%%%%%%%%%%%%%%%%
	% Conditions on BM
	%%%%%%%%%%%%%%%%%%%%%%%%%%%%%%%%%%%%%%%%
	We start with the former. 
	We first define the set $\mathcal{E}$ for the Brownian motion.
	Consider the set of positive rational numbers.
	For every $q \in \Rat^+$, 
	define 
	\begin{equation} \label{E: def l(q)}
		l(q) := \sup\{ \argmin_{s < q} \WtX(s) \}
	\end{equation}
	and
	\begin{equation} \label{E: def r(q)}
		r(q) := \inf\{ {s > q \; | \; \WtX(s) = \WtX(l(q))} \}.
	\end{equation}
	
	Now every rational number belongs to one excursion $(l(q), r(q))$,
	while one excursion contains multiple rational numbers.
	We define the set of excursions
	\begin{equation}
	\mathcal{E} := \bigcup_{q \in \Rat_+} \{ (l(q), r(q)) \}
	\end{equation}
	and note that it is countable.
	
	The following properties will be proven on a standard Brownian motion $W$,
	we later use Girsanov's theorem to apply them to $\WtX$.
	
	Consider two excursions $(l_1, r_2), (l_2, r_2)$ with $l_1 < l_2$.
	We want to show that, almost surely, for any two excursions $(l_1, r_2), (l_2, r_2)$ with $l_1 < l_2$
	\begin{equation} \label{E: W(l2) < W(l1)}
		W(l_2) = \min_{s \leq l_2} W(s) < W(l_1)
	\end{equation} 
	holds.
	If not, there exists an excursion $(l,r)$ such that another excursion begins immediately at $r$
	or that $W$ is constant for some time after $r$ and then starts a new excursion.
	Fix $q \in \Rat_+$. 
	By \eqref{E: def r(q)} $r(q)$ is a stopping time and by the strong Markov property of Brownian motion
	$W(s) - W(r(q))$ behaves like a standard Brownian motion for $s \geq r(q)$.
	By \cite[Theorem 2.8, p.38]{Morters.2010},
	$W$ has positive and negative values on any interval $[0,\delta]$ for $\delta > 0$ almost surely.
	Therefore for all $\delta > 0$ there exists $x \in (r(q), r(q) + \delta)$ with $W(x) < W(r(q))$.
	Thus, \eqref{E: W(l2) < W(l1)} holds almost surely for each $(l,r) = (l(q), r(q))$ and since $\Rat$ is countable,
	condition \eqref{E: f cond f(l1) > f(l2)} holds with probability $1$.
	
	The complement of all excursion is the set of intervals
	on which the Brownian motion is monotonously decreasing.
	By \cite[Theorem 1.22, p.18]{Morters.2010}, 
	$W$ is not monotonous on any interval $[a,b]$ with $0 \leq a < b < \infty$ almost surely.
	Therefore almost surely there is no interval in between any two excursions on which $W$ is monotonously decreasing
	and condition \eqref{E: f cond complement zero} holds almost surely.
	
	A standard Brownian motion $W$ satisfies the conditions of Lemma~\ref{L: Deterministic Lemma} almost surely,
	therefore $\WtX$ does so almost surely under $\tilde{\mathbb{P}}$,
	the equivalent probability measure defined in \eqref{E: Girsanov def P tilde},
	and by Girsanov's theorem (Theorem \ref{T: Girsanov}) likewise under $\Prob$.

	
	%%%%%%%%%%%%%%%%%%%%%%%%%%%%%%%%%%%%%%%%
	% Conditions on Zn
	%%%%%%%%%%%%%%%%%%%%%%%%%%%%%%%%%%%%%%%%
	We now show that conditions 
	\eqref{E: fn cond 3} to \eqref{E: fn cond 5}
	hold for the random walk $Z_n$ and $t_{n,i} = \n{-2}{3} \gamma_n(i)$.
	
	The breadth-first walk $\znt$ is a discrete process, therefore for all $n \in \Nat$ and $i \geq 2$:
	\begin{equation}
		\tn{i} - \tn{i-1} = \n{-2}{3}\gamma_n(i) - \n{-2}{3}\gamma_n(i-1) \geq \n{-2}{3} > 0.
	\end{equation}
	\fxnote{fix scaling?}
	By definition of $\gamma_n(i)$,
	$\lim_{i \rightarrow \infty} \gamma_n(i) = \infty$ for all $n$
	which establishes condition \eqref{E: fn cond 3}.
	
	The breadth-first walk attains a new minimum at the end of every component,
	which ensures \eqref{E: fn cond 4}.
	The difference between the levels of $\znt$ at the end of two consecutive components is always $1$,
	so 
	\begin{equation}
		\max_{i: \tn{i} \leq s_0}(\rzntX(\tn{i}) - \rzntX(\tn{i+1})) = \n{-1}{3} \xrightarrow{n \rightarrow \infty} 0
	\end{equation}
	for all $s_0 > 0$.
	
	
	%%%%%%%%%%%%%%%%%%%%%%%%%%%%%%%%%%%%%%%%
	% Completing the proof
	%%%%%%%%%%%%%%%%%%%%%%%%%%%%%%%%%%%%%%%%
	For almost all realizations $\rzntX(\omega')$ and $\WtX(\omega')$ 
	the processes and sets defined meet all conditions of Lemma~\ref{L: Deterministic Lemma}
	and we can establish the convergence
	\begin{equation}	
		\XinX_{\rzntX(\omega')} \rightarrow_v \XiX_{\WtX(\omega')}.
	\end{equation}
	This convergence holds almost surely, so
	\begin{equation}
		\XinX \rightarrow_{a.s.} \XiX
	\end{equation}
	with regard to the vague topology.
	Since almost sure convergence implies convergence in distribution we have
	\begin{equation}
		\XinX \rightarrow_d \XiX
	\end{equation}
	and therefore
	\begin{equation}
		\Xin \rightarrow_d \Xi
	\end{equation}
	with regard to the vague topology, which completes the proof.	
\end{proof}


%%%%%%%%%%%%%%%%%%%%%%%%%%%%%%%%%%%%%%%%%%%%%%%%%%%%%%%%%%%%
% What's still missing
%%%%%%%%%%%%%%%%%%%%%%%%%%%%%%%%%%%%%%%%%%%%%%%%%%%%%%%%%%%%

It remains to be shown that we need not consider the problem of a large components discovered by the breadth-first walk
or large excursions of the Brownian motion wandering off to infinity.

\section{Analysis of atypical excursions} \label{S: late excursions}
%%%%%%%%%%%%%%%%%%%%%%%%%%%%%%%%%%%%%%%%%%%%%%%%%%%%%%%%%%%%
% SECTION: Late excursions of Z_n
%%%%%%%%%%%%%%%%%%%%%%%%%%%%%%%%%%%%%%%%%%%%%%%%%%%%%%%%%%%%
From the previous section we assert that all excursions before time $C\n{2}{3}$, larger than $\delta\n{2}{3}$, for $C, \delta > 0$
eventually converge.
In this section we analyse the behaviour of $\rznt$ and its excursions after time $C$ in order to proof the convergence
of the whole ordered vector of component sizes and surplus edges.

\begin{lemma} \label{L: late excursions}
	Let 
	$\Prob(\ExcursionEvent)$ 
	be the probability that the breadth-first walk $\znt$ makes an excursion $\gamma$ of length 
	$|\gamma| > \delta\n{2}{3}$ starting after step $C\n{2}{3}$.
	Then for all $\epsilon>0$ and $\delta > 0$ exists $C>0$ such that $\sup_n \Prob(\ExcursionEvent) < \epsilon$. 
\end{lemma}
\begin{note}[Test]
	As mentioned before, in \cite[Lemma 9, p.826]{Aldous.1997} a similar statement is proven
	by citing bounds on the numbers of tree components, unicyclic components and complex components
	given in \cite{Luczak.1994}.
	We present a proof independent of existing random graph results.
\end{note}

\begin{proof}
	By the law of total expectation,
	\begin{equation} \label{E: P(ExcEvent) 1}
	\begin{aligned}
	\Prob(\ExcursionEvent) 
	&\leq \Exp{ \text{Number of excursions $\gamma: \; |\gamma| \geq \delta\n{2}{3},l(\gamma) \geq C\n{2}{3}$ } } \\
	&= \Exp{ \sum_{\gamma: \; l(\gamma) \geq C\n{2}{3}} \Ind{\{|\gamma| \geq \delta\n{2}{3} \} } } \\
	&\leq \Exp{ \sum_{\gamma: \; l(\gamma) \geq C\n{2}{3}} \frac{|\gamma|^2}{\delta^2\n{4}{3}} } \\
	&= \frac{1}{\delta^2\n{4}{3}} \Exp{ \sum_{\gamma: \; l(\gamma) \geq C\n{2}{3}}  |\gamma|^2 }.
	\end{aligned}
	\end{equation}
	
	Let $T$ be the time the last excursion starting before $C\n{2}{3}$ ends.
	The behaviour of the breadth-first walk after $T$ will be the same as the behaviour of a new walk on $\Gcal(n-T, \p)$.
	We write $\Ccal \in \Gcal$ to denote a component $\Ccal$ contained in the random graph $\Gcal$,
	and $|\Ccal|$ for its size.
	Since the notions of excursions of the breadth-first walk and components in the underlying graph are interchangeable,
	we can rewrite \eqref{E: P(ExcEvent) 1} as
	\begin{equation} \label{E: P(ExcEvent) 2}
	\begin{aligned}
	\Prob(\ExcursionEvent) 
	&\leq \frac{1}{\delta^2\n{4}{3}} \Exp{ \sum_{\Ccal \in \Gcal(n-T, \p)}  |\Ccal|^2  } \\
	&\leq \frac{1}{\delta^2\n{4}{3}} \Exp{ \sum_{\Ccal \in \Gcal(n-C\n{2}{3}, \p)}  |\Ccal|^2  } \\
	&\leq \frac{1}{\delta^2\n{4}{3}} \Exp{ \sum_{\Ccal \in \Gcal(n-C\n{2}{3}, \p)}  |\Ccal|^2 }.
	\end{aligned}
	\end{equation}
	
	For ease of notation we consider the graph $\Gcal(k,\p)$ and calculate
	\begin{equation} \label{E: P(ExcEvent) 3}
	\begin{aligned}
	\Exp{\sum_{\Ccal \in \Gcal(k, \p)} |\Ccal|^2 } 
	&= \Exp{ \sum_{\Ccal \in \Gcal(k, \p)} |\Ccal| \sum_{v \in \Ccal} 1} \\
	&= \Exp{ \sum_{\Ccal \in \Gcal(k, \p)} |\Ccal| \sum_{v \in \Gcal(k, \p)} \Ind{\{v \in \Ccal\}} } \\
	&= \sum_{v \in \Gcal(k, \p)} \Exp{ \sum_{\Ccal \in \Gcal(k, \p)} |\Ccal| \Ind{\{v \in \Ccal\}} } \\ 
	&= \sum_{v \in \Gcal(k, \p)} \Exp{|\Ccal(v)|} \\
	&= k \Exp{|\Ccal(1)|},
	\end{aligned}
	\end{equation}
	where $\Ccal(v)$ denotes the component containing the vertex $v$ and the last inequality stems from the interchangeability of the vertex labels.
	We will bound the expectation of the size of this component from above by a suitable branching process $(Y_i, \; i\geq 0)$.
	Starting at time $0$ with one vertex, we have $Y_0 = 1$.
	The number of children of this vertex is a $\Binom(k-1, \p)$ distributed random variable, $Y_1$.
	In the next step, each child-vertex will itself have children,
	each Binomially distributed on the remaining set of vertices with probability $ \p$.
	We compute 
	\begin{equation}
	\begin{aligned}
	Y_{2,1} &\sim \Binom(k-1-Y_1, \p), \\ 
	Y_{2,2} &\sim \Binom(k-1-Y_1-Y_{2,1}, \p), \\
	&\dots \\
	Y_{2,Y_1} &\sim \Binom(k-1-Y_1-Y_{2,1}-\dots-Y_{2,Y_1-1}, \p) \\
	Y_2 &= \sum_{i=1}^{Y_1} Y_{2,i}.
	\end{aligned}
	\end{equation}
	The size of the component will then be the total amount of explored vertices,
	which is the sum of all $Y_j$, $j \geq 0$.
	To provide an upper bound we consider the branching process where each amount of children is $\Binom(k, \p)$ distributed.
	Define the process as follows,
	\begin{equation}
	\begin{aligned}
	Z_0 &:= 1, \\
	Z_j &:= \sum_{i=1}^{Z_{j-1}} Z_{j,i},
	\end{aligned}
	\end{equation}
	where $Z_{j,i} \sim\Binom(k,p)$ for all $i,j \geq 1$.
	Then the process $(Y_i, \; i\geq 0)$ is stochastically dominated by 
	$(Z_i, \; i\geq 0)$ and
	\begin{equation}
	|\Ccal(1)| \leq_{\text{st.}} Z_0 + Z_1 + Z_2 + \dots
	\end{equation}
	which gives
	\begin{equation} \label{E: Exp C(1) <= sum Z}
		\Exp{|\Ccal(1)|} \leq \sum_{j=0}^{\infty}\Exp{Z_j}.
	\end{equation}	
	For $j\geq0$ we calculate the expectation of $Z_j$ by
	\begin{equation}
	\begin{aligned}
	\Exp{Z_j} 
	&= \Exp{Z_{j-1}}k \p \\
	&= \dots \\
	&= \Exp{Z_0}(k\p)^j \\
	&= (k\p)^j.
	\end{aligned}		
	\end{equation}
	Substituting in \eqref{E: Exp C(1) <= sum Z} gives
	\begin{equation} \label{E: Exp C(1) <= frac kp}
	\begin{aligned}
	\Exp{|\Ccal(1)|} 
	&\leq \sum_{j=0}^{\infty} (k \p)^j \\
	&\leq \frac{1}{(1- k\p)_+},
	\end{aligned}		
	\end{equation}
	where $(1- k\p)_+$ denotes the positive part of $1- k\p$.
	We continue the calculation in \eqref{E: P(ExcEvent) 2} using \eqref{E: P(ExcEvent) 3} and \eqref{E: Exp C(1) <= frac kp},
	which yields
	\begin{equation}
	\begin{aligned}
	\Prob(\ExcursionEvent) 
	&\leq \frac{n-C\n{2}{3}}{\delta^2 \n{4}{3}} \frac{1}{1-(n-C\n{2}{3}) \p} \\
	&= \frac{n-C\n{2}{3}}{\delta^2 \n{4}{3}} \frac{1}{1-(n-C\n{2}{3})(n^{-1} + \pp\n{-4}{3})} \\
	&= \frac{n-C\n{2}{3}}{\delta^2 \n{4}{3}} \frac{\n{1}{3}}{C-\pp+C\pp\n{-1}{3}} \\
	&\leq \frac{n}{\delta^2 \n{4}{3}} \frac{\n{1}{3}}{C-\pp+C\pp\n{-1}{3}} \\
	&= \delta^{-2} \frac{1}{C-\pp+C\pp\n{-1}{3}}.
	\end{aligned}
	\end{equation}
	For fixed $C$, $\delta$ and $\pp$
	this expression converges asymptotically to $\delta^{-2} \frac{1}{C-\pp}$ as $n \rightarrow \infty$.
	Therefore for all $\epsilon>0$ we can choose $C>0$ such that
	\begin{equation}
		\Prob(\ExcursionEvent) \leq \delta^{-2} \frac{1}{C-\pp+C\pp\n{-1}{3}} < \epsilon
	\end{equation}
	for all $n$. 
	This completes the proof.
\end{proof}

While this lemma provides enough information to prove the convergence with regard to the product topology of Folk Theorem~\ref{T: folk theorem},
Theorem~\ref{T: Main} requires additional information.
From \cite[Lemma 25, p.843]{Aldous.1997} we cite the following lemma on excursion lengths of Brownian motion.
Aldous' paper provides a complete proof using stochastic calculus methods.
\begin{lemma} \label{L: BM in ld}
	Let $\Gamma^{\pp}$ be the set of excursions of $\Bt$ and let $|\gamma|$ be the length of an excursion $\gamma$.
	Then
	\begin{equation}
		\Exp{\sum_{\gamma \in \Gamma^{\pp}} |\gamma|^2} < \infty.
	\end{equation}
\end{lemma}

Since $\sum_{\gamma \in \Gamma^{\pp}} |\gamma|^2$ is a convergent series almost surely,
for all $\epsilon > 0$ we can find $\delta > 0$ such that
\begin{equation} \label{E: gamma delta epsilon}
	\Exp{\sum_{|\gamma| < \delta} |\gamma|^2} < \epsilon.
\end{equation}
Then by Markov's inequality for all $\epsilon_0 > 0$ and $\epsilon_1 > 0$ there exists $\delta > 0$ such that
\begin{equation} \label{E: sum gamma < delta}
	\Prob(\sum_{|\gamma| < \delta} |\gamma|^2 > \epsilon_1) < \epsilon_0.
\end{equation}
An immediate corollary is the following equivalent of Lemma~\ref{L: late excursions} for the Brownian motion.

\begin{lemma} \label{L: late excursions gamma}
	Let $\Prob(\Event{\delta, C})$ be the probability that $\Bt$ makes an excursion $\gamma$ 
	of length $|\gamma| > \delta$ with $l(\gamma) > C$.
	Then for all $\epsilon > 0$ and $\delta > 0$ there exist $C > 0$ such that $\Prob(\Event{\delta, C}) < \epsilon$.
\end{lemma}	

We state and prove a result similar to \eqref{E: gamma delta epsilon} on excursion lengths of the rescaled breadth-first walk.
\begin{lemma} \label{L: sum rgamman}
	Denote by $\rgamman$ the excursions of $\rznt$. 
	Then for all $\epsilon > 0$ there exists $\delta > 0$ such that
	\begin{equation}
		\sup_n \Exp{\sum_{|\rgamman| < \delta} |\rgamman|^2} < \epsilon.
	\end{equation}
\end{lemma}
\begin{proof}
	We have
	\begin{equation}
		\Exp{\sum_{|\rgamman| < \delta} |\rgamman|^2} 
		= \Exp{\sum_{l(\rgamman) > C} |\rgamman|^2} 
		+ \Exp{\sum_{\substack{l(\rgamman) \leq C \\ |\rgamman| < \delta}} |\rgamman|^2}.
	\end{equation}
	Consider the second term:
	\begin{equation}
		\Exp{\sum_{\substack{l(\rgamman) \leq C \\ |\rgamman| < \delta}} |\rgamman|^2}
		= \Exp{\sum_{k = 0}^{\infty} \sum_{\substack{l(\rgamman) \leq C \\ |\rgamman| \in I_{\delta, k}}} |\rgamman|^2}
	\end{equation}
	where $I_{\delta, k} = [ \frac{\delta}{2^{k+1}}, \frac{\delta}{2^{k}} ]$.
	Then
	\begin{equation}
		\sum_{\substack{l(\rgamman) \leq C \\ |\rgamman| \in I_{\delta, k}}} |\rgamman|^2 
		\leq \left(\frac{\delta}{2^k}\right)^2 \frac{C}{\frac{\delta}{2^{k+1}}}
	\end{equation}
	and since this is now an entirely deterministic term we have
	\begin{equation}
	\begin{aligned}
	\Exp{\sum_{\substack{l(\rgamman) \leq C \\ |\rgamman| < \delta}} |\rgamman|^2} 
	&\leq \sum_{k = 0}^{\infty} \left(\frac{\delta}{2^k}\right)^2 \frac{C}{\frac{\delta}{2^{k+1}}} \\
	&= \sum_{k = 0}^{\infty} \frac{2C\delta}{2^k} \\
	&= 4C\delta.
	\end{aligned}
	\end{equation}
	By Lemma~\ref{L: late excursions} for all $\epsilon > 0$ we can choose $C > 0$ such that 
	\begin{equation}
		\Exp{ \sum_{l(\rgamman) > C} |\rgamman|^2 } < \epsilon / 2.
	\end{equation}
	For this $C$ choose $\delta > 0$ such that $4C\delta < \epsilon/2$.
	Then
	\begin{equation}
		\Exp{\sum_{|\rgamman| < \delta} |\rgamman|^2}  < \epsilon.
	\end{equation}
	Since $\delta$ and $C$ are independent of $n$ we have
	\begin{equation}
	\sup_n \Exp{\sum_{|\rgamman| < \delta} |\rgamman|^2} < \epsilon.
	\end{equation}
\end{proof}


\section{Finishing the proofs} \label{S: proofs}
%%%%%%%%%%%%%%%%%%%%%%%%%%%%%%%%%%%%%%%%%%%%%%%%%%%%%%%%%%%%
% SECTION: Finishing the proofs
%%%%%%%%%%%%%%%%%%%%%%%%%%%%%%%%%%%%%%%%%%%%%%%%%%%%%%%%%%%%
After all these technical considerations we are now ready to prove our main theorem.

\begin{proof}[Proof of Theorem~\ref{T: Main}]
	We first prove the convergence of component sizes and surplus edges to excursions and mark counts of $\Bt$ in the product topology
	and then the convergence of component sizes to Brownian excursions in the $\ld$ topology.
	
	\begin{proof}[Part 1]\renewcommand{\qedsymbol}{}
	We again define $\rzntX \sim \rznt$ and $\WtX \sim \Wt$ on the same probability space,
	where $\rzntX \rightarrow_{a.s.} \WtX$.
	Define
	\begin{equation} \label{E: def Yn}
		\Yn := ((|\rgamman_1|, \rsigman_1), (|\rgamman_2|, \rsigman_2), \dots)
	\end{equation}
	as the excursion lengths of $\rzntX$ and corresponding counts of $\rnntX$,
	as well as
	\begin{equation} \label{E: def Y}
		\Y := ((|\gamma_1|, \mu(\gamma_1)), (|\gamma_2|, \mu(\gamma_2)), \dots)
	\end{equation}
	for the lengths of excursions and corresponding mark counts of $\BtX$, each ordered by decreasing excursion length.
	Denote by 
	\begin{equation} \label{E: def Yk Ynk}
		\begin{aligned}
		\Ynk &:= ((|\rgamman_1|, \rsigman_1), \dots, (|\rgamman_k|, \rsigman_k)) \\
		\Yk  &:= ((|\gamma_1|, \mu(\gamma_1)), \dots, (|\gamma_k|, \mu(\gamma_k)))
		\end{aligned}
	\end{equation}
	the vector of the first $k$ components of the corresponding vector.
	We want to show that for all $k \in \Nat$ we have
	\begin{equation}
		\Ynk \rightarrow_d \Yk.
	\end{equation}
	
	Fix $k \in \Nat$. 
	Since $\Yk$ is a vector of finite length, for all $\epsilon > 0$ we can find $\delta > 0$ such that
	\begin{equation} \label{E: folk theorem cond bm1}
		\Prob(|\gamma_k| < \delta) < \epsilon.
	\end{equation}
	For a realization of the processes $\rzntX(\omega)$ and $\WtX(\omega)$ we know from Lemma~\ref{L: late excursions}
	that for all $\epsilon> 0$ and $\delta > 0$ there exists $C > 0$ such that
	\begin{equation} \label{E: folk theorem cond znt}
		\sup_n \Prob( \exists \rgamman: \; l(\rgamman) > C, |\rgamman| > \delta ) < \epsilon
	\end{equation}
	and Lemma~\ref{L: late excursions gamma} gives
	\begin{equation} \label{E: folk theorem cond bm2}
		\Prob(\exists \gamma: \; l(\gamma) > C, |\gamma| > \delta) < \epsilon.
	\end{equation}
	Since $\rzntX \rightarrow_{a.s.} \WtX$ for all $\beta > 0$, there exists $N \in \Nat$ such that for all $n \geq N$ we have
	\begin{equation} \label{E: folk theorem cond sup}
		\Prob( \sup_{s \in [0, C]} |\rzntX(s) - \WtX(s)| > \beta ) < \epsilon.
	\end{equation}
	
	We show that $\YnkX \rightarrow_p \YkX$, i.e. for all $\alpha > 0$ we have
	\begin{equation}
		\Prob( d(\YnkX, \YX) > \alpha) \rightarrow 0
	\end{equation}
	as $n \rightarrow \infty$.
	Denote by $\Event{\YkX}, \Event{\rgamman}, \Event{\gamma}, \Event{\beta}$ the events in 
	\eqref{E: folk theorem cond bm1}, \eqref{E: folk theorem cond znt}, \eqref{E: folk theorem cond bm2} and \eqref{E: folk theorem cond sup} respectively.
	Then
	\begin{equation}
		\begin{aligned}
		\Prob( d(\YnkX, \YX) > \alpha) 
		&\leq \Prob( d(\YnkX, \YX) > \alpha \cond \neg\Event{\YkX}, \neg\Event{\rgamman}, \neg\Event{\gamma}, \neg\Event{\beta}) \\
		&\quad + \Prob(\Event{\YkX}) + \Prob(\Event{\rgamman}) + \Prob(\Event{\gamma}) + \Prob(\Event{\beta}). 
		\end{aligned}
	\end{equation}
	The conditions in the first probability assert that there are no large excursions of $\rzntX$ or $\WtX$ after $C$,
	that the $k$-th excursion of $\WtX$ is large and $\rzntX$ and $\WtX$ are close.
	Since excursions of $\rzntX$ and $\WtX$ eventually match up, 
	all $k$ relevant excursions must happen before $C$.
	With shrinking distance between $\rzntX$ and $\WtX$ the distance between excursion lengths must become smaller as well
	and for sufficiently large $n$ the number of surplus edges in $\YnkX$ coincides exactly with number of marks in $\YkX$.
	Therefore for all $\alpha > 0$ we can find $\beta = \beta(\alpha)$ such that
	\begin{equation}
		\Prob( d(\YnkX, \YX) > \alpha \cond \neg\Event{\YkX}, \neg\Event{\rgamman}, \neg\Event{\gamma}, \neg\Event{\beta}) = 0.
	\end{equation}
	
	Then for all $\epsilon > 0$ we choose $\delta > 0$ such that $\Prob(\Event{\YkX}) < \epsilon / 4$,
	for these $\delta$ and $\epsilon$ we find $C > 0$ such that $\Prob(\Event{\rgamman}) < \epsilon / 4$ and $\Prob(\Event{\gamma}) < \epsilon / 4$
	and finally $N \in \Nat$ such that $\Prob(\Event{\beta}) < \epsilon$ for all $n \geq N$.
	Then
	\begin{equation}
		\Prob( d(\YnkX, \YX) > \alpha) \leq \epsilon
	\end{equation}
	which proves $\YnkX \rightarrow_p \YkX$.
	Now convergence in probability implies convergence in distribution which completes the proof of the first and second statement of Theorem~\ref{T: Main}.
	\end{proof}


	\begin{proof}[Part 2]\renewcommand{\qedsymbol}{}
	The only thing left to show is that the convergence $\n{-2}{3} \Cntbold \rightarrow_p \Ctbold$ holds with respect to the $\ld$~topology.
	
	For $\delta > 0$ and $\epsilon_1 > 0$, denote by $\Event{\sum\gamma}$ the event that $\sum_{|\gamma| < \delta}|\gamma|^2 > \epsilon_1$.
	By Lemma~\ref{L: BM in ld} for all $\epsilon > 0$ there exist $\delta > 0$ and $\epsilon_1 > 0$ such that
	\begin{equation}
	\Prob(\Event{\sum\gamma}) = \Prob( \sum_{|\gamma| < \delta}|\gamma|^2 > \epsilon_1 ) < \epsilon.
	\end{equation}
	
	Similarly, denote by $\Event{\sum\rgamman}$ the event that $\sum_{|\rgamman| < \delta}|\rgamman|^2 > \epsilon_2$.
	By Lemma~\ref{L: sum rgamman} for all $\epsilon > 0$ there exist $\delta > 0$ and $\epsilon_2 > 0$ such that
	\begin{equation}
	\Prob(\Event{\sum\rgamman}) = \Prob( \sum_{|\rgamman| < \delta}|\rgamman|^2 > \epsilon_2 ) < \epsilon.
	\end{equation}
	
	Fix $\epsilon > 0$. 
	Then for all $\alpha > 0$ we have 
	\begin{equation} \label{E: main proof step 1}
	\begin{aligned}
	&\Prob( d_{2}(\YnX, \YX) > \alpha) \\
	&\quad\leq \Prob( d_{2}(\YnX, \YX) > \alpha \cond 
	\neg\Event{\sum\rgamman}, \neg\Event{\sum\gamma}, \neg\Event{\gamma}, \neg\Event{\rgamman}, \neg\Event{\beta}) \\
	&\quad\quad + \Prob(\Event{\sum\rgamman}) + \Prob(\Event{\sum\gamma}) + \Prob(\Event{\gamma}) + \Prob(\Event{\rgamman}) + \Prob(\Event{\beta}).
	\end{aligned}
	\end{equation}
	Choose $\delta > 0$ and $C > 0$ such that $\Prob(\Event{\gamma}) < \epsilon / 5$ and $\Prob(\Event{\rgamman}) < \epsilon / 5$.
	For this $\delta$, choose $\epsilon_1, \epsilon_2 > 0$ with $\epsilon_1 + \epsilon_2 < \alpha$
	such that $\Prob(\Event{\sum\rgamman}) < \epsilon / 5$ and $\Prob(\Event{\sum\gamma}) < \epsilon / 5$.
	Now consider the conditioned probability in \eqref{E: main proof step 1}.
	The distance between all entries of $\YnX$ and $\YX$ not starting before $C$ or with size smaller than $\delta$ 
	will be at most $\epsilon_1 + \epsilon_2$.
	As $\beta \rightarrow 0$ the distance of all other excursions, say $d_{\beta}$, vanishes as well.
	Therefore choose $\beta > 0$ such that $\Prob(\Event{\beta}) < \epsilon / 5$ and $\epsilon_1 + \epsilon_2 + d_{\beta} < \alpha$.
	Then
	\begin{equation}
	\Prob( d_{2}(\YnX, \YX) > \alpha \cond 
	\neg\Event{\sum\rgamman}, \neg\Event{\sum\gamma}, \neg\Event{\gamma}, \neg\Event{\rgamman}, \Event{\beta}) = 0
	\end{equation}
	and 
	\begin{math}
	\Prob( d_{2}(\YnX, \YX) > \alpha) < \epsilon.
	\end{math}
	Therefore $\YnX \rightarrow_p \YX$ which implies $\Yn \rightarrow_d \Y$ with regard to the $\ld$~topology.
	\end{proof}

	
\end{proof}


