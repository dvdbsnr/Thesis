% Chapter 4: Convergence of component sizes
% Contains:
%   Explanation on what's still missing
%   The deterministic lemma (Lemma 7)
%   Girsanov Theorem
%   Applying Lemma 7 (Lemma 8)
%   Graph theory approach:
%     Lemma 9
%   l2 approach:
%     Definitions on l2
%     Deterministic lemma (Lemma 14)
%     Proposition 15
%     Applying Prop 15 to the bf-walk

\chapter{Convergence of component sizes}
\fxnote{Update title.}

%%%%%%%%%%%%%%%%%%%%%%%%%%%%%%%%%%%%%%%%%%%%%%%%%%%%%%%%%%%%
% What's still missing
%%%%%%%%%%%%%%%%%%%%%%%%%%%%%%%%%%%%%%%%%%%%%%%%%%%%%%%%%%%%


%%%%%%%%%%%%%%%%%%%%%%%%%%%%%%%%%%%%%%%%%%%%%%%%%%%%%%%%%%%%
% Deterministic Lemma 7: Statement and Note
%%%%%%%%%%%%%%%%%%%%%%%%%%%%%%%%%%%%%%%%%%%%%%%%%%%%%%%%%%%%
\begin{lemma} \label{L: Deterministic Lemma}
	Let $f:[0, \infty) \rightarrow \Real$ be a continuous function. 
	Let $\mathcal{E}$ be the set of non-empty intervals 
	$e=(l,r) \subset \Real_{\geq 0}$
	such that
	\begin{equation} \label{E: f cond 1}
	f(r) = f(l) = \min_{s \leq l} f(s),
	\end{equation}
	\begin{equation} \label{E: f cond 2}
	f(l) < f(s) \quad \forall l < s < r.
	\end{equation}
	Define $\Xi := \lbrace (l, r-l) \; | \; (l, r) \in \mathcal{E} \rbrace$.
	
	Suppose that for intervals $e_1, e_2 \in \mathcal{E}$ we have 
	\begin{equation} \label{E: f cond f(l1) > f(l2)}
	f(l_1) > f(l_2)
	\end{equation}
	and the complement of $\cup_{e \in \mathcal{E}} (l,r)$ has Lebesgue measure zero,
	\begin{equation} \label{E: f cond complement zero}
	\mu \left( \left( \cup_{e \in \mathcal{E}} (l,r) \right)^c\right) = 0.
	\end{equation}
	
	Let $f_n \rightarrow f$ as $n \rightarrow \infty$ uniformly on bounded intervals.
	Suppose $(\tn{i}, i \geq 1)$ satisfy the following conditions:
	
	\begin{equation} \label{E: fn cond 3}
	0 = \tn{1} < \tn{2} < ... \; \text{and} \; \lim_{i \rightarrow \infty} \tn{i} = \infty,
	\end{equation}
	\begin{equation} \label{E: fn cond 4}
	f_n(\tn{i}) = \min_{u \leq \tn{i}}f_n(u), 
	\end{equation}
	\begin{equation} \label{E: fn cond 5}
	\max_{i: \tn{i} \leq s_0}(f_n(\tn{i}) - f_n(\tn{i+1})) \rightarrow 0 \; \text{as} \; n \rightarrow \infty, \; \text{for all} \; s_0 \leq \infty.
	\end{equation}
	
	Define $\Xin := \lbrace (\tn{i}, \tn{i+1} - \tn{i}) \; | \; i \geq 1 \rbrace$.
	Then $\Xin \rightarrow \Xi$ as $n \rightarrow \infty$.
\end{lemma}
\fxnote{Conditions in Lemma -> Ask Sapozhnikov}

\begin{note}
	The convergence
	$\Xin \rightarrow_d \Xi$
	is to be interpreted as the vague convergence of counting measures,
	which means that for all continuous functions
	$f: \IntTimesInt \rightarrow \Real$
	with compact support,
	\begin{equation}
		\Exp{f(\Xin)} \rightarrow \Exp{f(\Xi)}.
	\end{equation}
	By general theory,
	see for example \cite{Kallenberg1990},
	\fxnote{Add real reference here}
	this is equivalent to the convergence
	\begin{equation}
	\Xin(C) \rightarrow \Xi(C)
	\end{equation}
	for all compact subsets of
	$\IntTimesInt$,
	where $\Xi(\partial C) = 0$.
	In this case, a compact subset is a pair of closed intervals
	$[T_1, T_2] \times [d_1, d_2]$,
	with $T_1, T_2 \geq 0$
	and $d_1, d_2 > 0$.
	The condition on the measure of the boundary means,
	that the limit process must not have any excursions starting exactly at
	$T_1$ or $T_2$,
	or any excursions of length exactly $d_1$ or $d_2$.
	An exception is the case of $T_1=0$.
	Since the domain of $f$ starts at $0$,
	and the boundary condition is needed to prevent the case of points in $\Xin$ converging to some point on the boundary "from outside",
	\fxnote{This is not really a boundary condition.}
	\fxnote{Are "" around "from outside" needed?}
	\fxnote{This is not a nice sentence in general.}
	we do not need to consider $0$ as part of the boundary of any interval 
	$[0, T_2] \times [d_1, d_2]$.
\end{note}
\fxnote{Does this fit in the note? 
	General stuff about vague convergence should not be in the proof.
	Maybe vague convergence before the Lemma, interval stuff in the note.}


%%%%%%%%%%%%%%%%%%%%%%%%%%%%%%%%%%%%%%%%%%%%%%%%%%%%%%%%%%%%
% Deterministic Lemma 7: Proof
%%%%%%%%%%%%%%%%%%%%%%%%%%%%%%%%%%%%%%%%%%%%%%%%%%%%%%%%%%%%
\begin{proof}
%	Structure:
%	- What does the definition of \Xi and \Xin mean?
%	- Detail on \Xin and conditions
%   First part: \Xi in \Xin
%   (For every f exc, there is an fn exc)
%	- Facts about f:
%	 - It goes up to the left and right of l
%	 - It goes down to the right of r? (Or later)
%	 - eps/delta stuff
%	- Arch of fn
%	- Definition ln(x)
%	- ln(x) in [l-e, l+e]
%	- Definition rn(x)
%	- rn(x) in [r-e, r+e]
We begin by elaborating what the conditions
\eqref{E: f cond 1} to \eqref{E: fn cond 5}
mean for the points in
$\Xi$ and $\Xin$.
\fxfatal{Some stuff about excursions goes here.}

To prove the Lemma, 
we fix some bounded subset of
$\IntTimesInt$,
$C := [T_1, T_2] \times [d_1, d_2]$,
and show that, for sufficiently high $n$,
$\Xin(C) = \Xi(C)$,
that is,
there are exactly as many excursions of $f_n$ starting in
$(T_1, T_2)$, 
with length in $(d_1, d_2)$,
as similar excursions of the limit function $f$.

We will first show that every excursion of $f$ is eventually matched by some excursion of $f_n$,
then show that there can not be any more excursion of $f_n$ of sufficient length.


%%%%%%%%%%%%%%%%%%%%%%%%%%%%%%
% Proof Part 1: \Xi in \Xin
%%%%%%%%%%%%%%%%%%%%%%%%%%%%%%
\begin{proofpart}[$\Xin(C) \subseteq \Xi(C)$] \label{PP: Lemma 7 1}

Let us first establish some facts about excursions of the limit function $f$.
In the interval $[T_1, T_2]$,
there can only be a finite number of excursion starting with length of at least $d_1$,
at most $T_1/d_1$.
\fxnote{Fix this sentence!}
Let
$ \mathcal{E}^* := \{ (l_i, r_i) \; | \; i=1, \dots, k \} $
be the set of these excursions.
Consider a fix
$(l,r) \in \mathcal{E}^*$.
\fxnote{Some figure showing an excusion goes here probably.}
Since $l-r > d_1$,
we can find an $\epsilon > 0$
such that the length of the interval
$[l+\epsilon, r-\epsilon]$
is still greater than $d_1$.

We can also find an $\epsilon > 0$ sufficiently small,
so that for every
$x \in [0, l-\epsilon] \cup [l+\epsilon, r-\epsilon]$,
\begin{equation} \label{E: fx > fl + delta}
f(x) > f(l) + \delta
\end{equation}
holds for some $\delta > 0$.

Assume this does not hold to the left of the $\epsilon$-neighbourhood of $l$.
Then there exists some $x \in [0, l-\epsilon]$
for which $f(x) \leq f(l) + \delta$ for every $\delta > 0$.
This implies $f(x) \leq f(l)$,
and by condition \eqref{E: f cond 1},
$f(l) = \min_{u \leq l}f(u)$,
so $f(x) = f(l)$.
That leaves two possibilities:
First, $f$ is constant on the interval $[x,l]$.
But this were an interval of non-zero length without any excursions on it,
a contradiction to \eqref{E: f cond complement zero}.
Second, there is an interval
$[x', l']$, with $x\leq x'<l' \leq l$,
such that $f(y) > f(x)$ for all $y \in (x',l')$.
That makes $(x',l')$ another excursion in $\mathcal{E}$,
but $f(x') = f(l)$,
which is a contradiction to condition \eqref{E: f cond f(l1) > f(l2)}.

Using the same logic,
assuming \eqref{E: fx > fl + delta} does not hold in $[l + \epsilon, r- \epsilon]$
leads to a point $x \in [l + \epsilon, r- \epsilon]$
such that $f(x) \leq f(l)$,
which is a contradiction to condition \eqref{E: f cond 2}.

Now consider the behaviour of $f$ on $[r, r + \epsilon]$.
As previously stated, 
condition \eqref{E: f cond complement zero} prevents $f$ from being constant.
If $f(x) \geq f(r)$ for all $x \in [r, r + \epsilon]$,
there would be another excursion that contradicts \eqref{E: f cond f(l1) > f(l2)}.
So there must exist an $r^* \in (r, r + \epsilon]$ with $f(r^*) < f(r)$.
Let $\delta^* := \min\{ \delta, f(r) - f(r^*) \}$,
where $\delta$ is the constant used in the discussion of
$[0, l-\epsilon]$ and $[l+\epsilon, r-\epsilon]$ above.
\fxnote{delta used in the discussion above is not nice.}


We now take a look at $f_n$.
Fix an $x^* \in [l+\epsilon, r-\epsilon]$.
We will now show that there exist points
$l_n(x^*) \in [l-\epsilon, l+\epsilon]$
and 
$r_n(x^*) \in [r-\epsilon, r+\epsilon]$
for which conditions \eqref{E: fn cond 3} to \eqref{E: fn cond 5} hold.

Define $l_n(x^*) := \argmin_{u \leq x} f_n(u)$.
\fxerror{Make ln(x) := sup(argmin(...))}
By the uniform convergence $f_n \rightarrow f$,
we can find an $N \in \Nat$,
such that $|f_n(x) - f(x)| < \delta* / 3$ for all
$x \in [T_1, T_2]$ and $n \geq N$.
So for every point $x \in [0, l - \epsilon]\cup[l + \epsilon, r - \epsilon]$,
\begin{equation} \label{E: fn(x) > fn(l)}
f_n(x) > f(x) - \delta^* / 3 > f(l) + \delta^* - \delta^* / 3 > f(l) + \delta^* / 3 > f_n(l).
\end{equation}
\fxnote{Align these.}
So $f_n$ takes its minimum over 
$[0, l - \epsilon]\cup[l + \epsilon, r - \epsilon]$
at $l$ or somewhere in $[l-\epsilon, l+\epsilon]$.
\fxnote{Maybe give the interval a name.}
So $l_n(x^*) \in [l-\epsilon, l+\epsilon]$ and
$l_n(x^^*) = \min_{u \leq l_n(x^*)} f_n(u)$.

Now define 
$r_n(x^*) := \inf \{ x > x^* \; | \; f_n(x) = l_n(x^*) \}$.
By \eqref{E: fn(x) > fn(l)}, the function $f_n$ can not reach its past minimum $l_n(x^*)$
before $r-\epsilon$.
As previously discussed,
there exists a $r^* \in (r, r+\epsilon]$ such that $f(r) - f(r^*) \geq \delta^*$.
Now
\begin{equation}
l_n(x) > l(x) - \delta^* / 3 = f(r) - \delta^* / 3 \geq f(r^*) + \delta^* - \delta^* / 3 > f_n(r^*),
\end{equation}
\fxnote{Align these.}
which implies that $r_n(x^*)$ must be smaller than $r^*$,
thus $r_n(x^*) \in [r - \epsilon, r+ \epsilon]$.
Since $l_n(x) = \min_{u \leq l_n(x^*)} f_n(u)$
and $r_n(x^*)$ is the first time this previous minimum is reached,
$r_n(x^*) = \min_{u \leq l_n(x^*)} f_n(u)$.

\fxerror{Here: What is a condition on fn? What must be shown?}

Since $l - r + 2\epsilon > d_1$,
we have found $\tn{i} = l_n(x^*)$, $\tn{i+1} = r_n(x^*)$,
such that $(\tn{i}, \tn{i+1} - \tn{i}) \in \Xin \cap C$
for all $n$ greater than some $N_i = N \in \Nat$.
Now let $N^* := \max\{N_1, \dots, N_k\}$ 
and every excursion of $f$ in $C$ is matched by an excursion of $f_n$ in $C$,
with $n \geq N^*$.
So eventually $\Xi (C) \leq \Xin(C)$.
\end{proofpart}


%%%%%%%%%%%%%%%%%%%%%%%%%%%%%%
% Proof Part 2: \Xin in \Xi
%%%%%%%%%%%%%%%%%%%%%%%%%%%%%%
\begin{proofpart}[$\Xi(C) \subseteq \Xin(C)$] \label{PP: Lemma 7 2}

Now that every excursion of $f$ is matched, 
we need to show that there can not exist any additional large excursion of $f_n$.
Considering the fact that excursions can not overlap,
the only possibility for an additional large excursion is the space between two large excursions.
\fxnote{This is quite obvious.}
For a pair of large excursions,
$(l_i, r_i)$ and $(l_{i+1}, r_{i+1})$,
there is only enough space in between them if
$r_i - l_{i+1} > d_1$.
Consider such an interval $[r,l]$ of length greater than $d_1$.
By condition \eqref{E: f cond complement zero} the space must be filled with smaller excursions of $f$.
There is an at most countable number of these,
let $\mathcal{E^*} := \{ (l_i, r_i) \; | \; i \in \Nat \}$
be the set of such excursions starting in $(r,l)$ with length less than or equal to $d_1$.
We know
\begin{equation}
\sum_{i=1}^{\infty} r_i - l_i = l - r > d_1,
\end{equation}
so we can choose a finite set of excursions
$\{ (l_i, r_i) \; | \; i = 1, \dots, k \}$
such that, 
if we exclude these from the interval $[r,l]$, 
the space remaining is less than $d_1$:
\fxerror{This is the first : ! Delete it?}
\begin{equation}
l-r - \sum_{i=1}^{k} r_i - l_i < d_1.
\end{equation}
Let $d^* < \min \{ r_i - l_i \; | \; i = 1, \dots, k \}$
and apply the logic of Part \ref{PP: Lemma 7 1} to the compact set
$[r, l] \times [d^*, d_1]$.
\fxnote{Length d1 or d2? Is there a problem with excursions of exactly length d1?}
For sufficiently large $n$, 
every of these $k$ excursions of $f$ will be matched with an excursion of $f_n$,
so that there will be no space left for a large excursion of $f_n$ in $[r,l]$.
Applying this logic to every one if the finitely many gaps between excursions of $f$,
we see that there can not exist more large excursions of $f_n$ already matching $f$.
\fxnote{There is half a sentence missing here.}
Thus $\Xin(C) \leq \Xi(C)$ for sufficiently large $n$,
which completes the proof of Lemma \ref{L: Deterministic Lemma}.
\end{proofpart}

	
\end{proof}


%%%%%%%%%%%%%%%%%%%%%%%%%%%%%%%%%%%%%%%%%%%%%%%%%%%%%%%%%%%%
% Lemma 8: Statement
%%%%%%%%%%%%%%%%%%%%%%%%%%%%%%%%%%%%%%%%%%%%%%%%%%%%%%%%%%%%
The following Lemma will now link $Z_n$ and $\Wt$ in the language of Lemma \ref{L: Deterministic Lemma}
Define $\gamma (n, i) \in \{1, \dots, n\}$ to be the index that makes
$v(\gamma(n,i))$ is the last vertex of the $i-1$-st component encountered by the breadth-first walk $Z_n$.
\fxnote{Text clunky here.}
Let $\Ci{n,i}$ be the size of the $i$-th component.
\fxnote{Fix i indices}

\begin{lemma} \label{L: Lemma 8}
	Let $\Xi$ be the point process with points corresponding to excursions of $\Bt$,
	\begin{equation} \label{E: Lemma 8 def Xi}
	\Xi := \{ (l(\gamma), |\gamma|) \; | \; \gamma \; \text{excursion of} \; \Bt \}.
	\end{equation}
	Let $\Xin$ be the point process with points corresponding to excursions of the breadth-first walk
	\begin{equation} \label{E: Lemma 8 def Xin}
	\Xin := \{ ( \n{-2}{3} \gamma(n,i), \n{-2}{3} \Ci{n,i} ) \; | \; i \geq 1 \}.
	\end{equation}
	Then $\Xin \rightarrow \Xi$ as $n \rightarrow \infty$.
\end{lemma}
\fxnote{Try not to repeat the same sentence twice here.}

Before proving this Lemma, we need to state Girsanov's Theorem, 
which will enable us to prove certain properties of the Brownian motion with drift.
\fxnote{Explanation here or at the end?}


%%%%%%%%%%%%%%%%%%%%%%%%%%%%%%%%%%%%%%%%%%%%%%%%%%%%%%%%%%%%
% Girsanov Theorem
%%%%%%%%%%%%%%%%%%%%%%%%%%%%%%%%%%%%%%%%%%%%%%%%%%%%%%%%%%%%
We state the theorem as in \cite[Theorem 5.2.3]{Shreve2004}
\begin{theorem}[Girsanov] \label{T: Girsanov}
	Let $W(t), 0 \leq t \leq T$, be a Brownian motion on a probability space
	$(\Omega, \mathcal{F}, \mathbb{P})$.
	Define
	\begin{align}
	\tilde{W}(t) &:= W(t) + \int_0^t \Theta(u)du, \label{E: Girsanov def W tilde} \\ 
	Z(t) &:= \exp \left\{ -\int_{0}^{t} \Theta(u) dW(u) - \frac{1}{2} \int_0^t \Theta^2(u)du \right\}. \label{E: Girsanov def Z}
	\end{align}
	Assume that
	\begin{equation} \label{E: Girsanov cond Theta}
	\ExpBig{ \int_{0}^{T} \Theta^2(u) Z^2(u)du  } < \infty.
	\end{equation}
	Set the random variable $Z=Z(T)$. Then $\Exp{Z(T)} = 1$ and $\mathbb{Q}$, defined by
	\begin{equation} \label{E: Girsanov def Q}
	\mathbb{Q}(A) := \int_A Z(\omega) d\mathbb{P}(\omega), \; \text{for all} \; A \in \mathcal{F},
	\end{equation}
	is a probability measure under which the process 
	$\tilde{W}(t)$, $0 \leq t \leq T$, 
	is a standard Brownian motion.
\end{theorem}
\fxerror{Check definition of Z=Z(T) in other references!}
\fxfatal{Brownian Motion -> Brownian motion in whole document!}
\fxerror{Q are already the rational numbers.}
\fxfatal{E(Z) = 1 is not a result, but a condition!}

We can apply this Theorem to the Brownian motion with drift $W^t$ as follows:
$\Wt(s) = W(s) + ts - \frac{1}{2} s^2 = W(s) + \int_0^s (t-u)du $.
Since $\Theta(u) = t-u$ obviously satisfies \eqref{E: Girsanov cond Theta} for $T < \infty$,
$\Wt$ is a standard Brownian motion under the probability measure $\mathbb{Q}$, 
as defined in \eqref{E: Girsanov def Q}. 
From it's definition in \eqref{E: Girsanov def Z}, $Z(T) > 0$ with probability $1$.
Thus, for all sets $A \in \mathcal{F}$,
\begin{equation} \label{E: P0 = Q0}
\mathbb{P}(A) = 0 \iff \mathbb{Q}(A) = 0.
\end{equation}
This means, that events that happen with probability $0$ under the measure $\mathbb{Q}$ also happen with probability $0$ under $\mathbb{P}$.
\fxerror{That's literally what it means. Some more interpretation might be nice.}
We will use this result to proof certain properties of the Brownian motion with drift,
by showing that these properties hold for the standard Brownian motion with probability $1$ and changing the measure.
\fxerror{Explain: "Changing the measure." Or fix this sentence after proof is complete.}


%%%%%%%%%%%%%%%%%%%%%%%%%%%%%%%%%%%%%%%%%%%%%%%%%%%%%%%%%%%%
% Lemma 8: Proof
%%%%%%%%%%%%%%%%%%%%%%%%%%%%%%%%%%%%%%%%%%%%%%%%%%%%%%%%%%%%
\begin{proof}[Proof of Lemma \ref{L: Lemma 8}]
	By the definition of $\Bt$, 
	$\Xi$ is the $\Xi$ in Lemma \ref{L: Deterministic Lemma} with $f = \Wt$.
	Taking $f_n$ to be $\bar{Z}_n$ and defining $\tn{i} = \n{-2}{3} \gamma(n,i)$,
	$\Xin$ is the $\Xin$ in Lemma \ref{L: Deterministic Lemma}.
	
	By Theorem \ref{T: Z -> W} we know $\bar{Z}_n \rightarrow_d \Wt$.
	\fxnote{Add this Theorem}
	By the Skorohod representation theorem,
	see \cite{Billingsley2009}, 
	\fxnote{Add reference.}
	\fxnote{Explain Skorohod.}
	we can construct a probability space 
	\fxnote{Can we construct one? Or does one just exist?}
	$(\Omega, F, P)$ 
	\fxnote{Add real probability space.}
	on which exist random variables
	$\bar{Y}_n$ and $V^t$,
	distributed as 
	$\bar{Z}_n$ and $\Wt$ respectively,
	that converge almost surely.
	Almost sure convergence implies that,
	with probability $1$,
	\begin{equation}
	d(\bar{Y}_n, V^t) = \sup_{x \in \Real}|\bar{Y}_n(x) - V^t(x)| \rightarrow 0,
	\end{equation}
	as $n \rightarrow \infty$.
	So $\bar{Y}_n(\omega)$ converges to $V^t(\omega)$ uniformly,
	especially on bounded intervals.
	\fxnote{Not "especially" on bounded intervals.}
	We still need to show that conditions \eqref{E: f cond 1} to \eqref{E: f cond complement zero} hold for $\Wt$
	and conditions \eqref{E: fn cond 3} to \eqref{E: fn cond 5} hold for the breadth-first walk.
	
	
	%%%%%%%%%%%%%%%%%%%%%%%%%%%%%%%%%%%%%%%%
	% Conditions on BM
	%%%%%%%%%%%%%%%%%%%%%%%%%%%%%%%%%%%%%%%%
	We start with the former. 
	We first define the set $\mathcal{E}$ for the Brownian motion.
	Consider the set of positive rational numbers.
	For every $q \in \Rat^+$, 
	set $l(q) := \sup( \underset{s \leq q}{\argmin} \Wt(s) ) $ 
	and $r(q) := \inf( {s \in \Real \; | \; \Wt(s) = l(q)} ) $.
	\fxerror{In this definition, if q is a minimum, l(q) = r(q) = q}
	Now every rational number belongs to one excursion $(l(q), r(q))$,
	while one excursion contains multiple rational numbers.
	We define the set of excursions
	\begin{equation}
	\mathcal{E} := \bigcup_{q \in \Rat^+} \{ (l(q), r(q)) \}
	\end{equation}
	and note that it is countable.
	\fxnote{When do we need countability? For the set definition?}
	The following properties will be proven on the standard Brownian motion $W$,
	we later take the step to $\Wt$.
	\fxnote{"Take the step" - Lame!}
	
	We first make sure that every interval $[l_i, r_i]$ is in fact non-trivial and condition \eqref{E: f cond 2} is met,
	since the definition above permits the case $l(q) = r(q) = q$.
	\fxerror{OR: Fix the definition. -> Talk to Sapozhnikov?}
	\fxerror{Here goes the proof of that. Some stuff about rational numbers.}
	
	The complement of all excursion is the set of intervals,
	on which the Brownian motion is monotonously decreasing.
	But, see \cite[Theorem 1.22]{Peres2008}, with probability $1$,
	$W$ is not monotonous on any interval $[a,b]$ with $0 < a < b < \infty$.
	\fxerror{Is that enough to prove that there is no monotonous interval?}
	
	These properties hold with probability $1$ for the standard Brownian motion.
	So they hold with probability $1$ for $\Wt$ under $\mathbb{Q}$
	and, by \eqref{E: P0 = Q0}, 
	almost surely for $\Wt$ under the original measure $\mathbb{P}$.
	\fxnote{"probability 1" - looks weird!}
	
	
	%%%%%%%%%%%%%%%%%%%%%%%%%%%%%%%%%%%%%%%%
	% Conditions on Zn
	%%%%%%%%%%%%%%%%%%%%%%%%%%%%%%%%%%%%%%%%
	We now show that conditions 
	\eqref{E: fn cond 3} to \eqref{E: fn cond 5}
	hold for the random walk $Z_n$ and $\Xin$
	as defined in \eqref{E: Lemma 8 def Xin}.
	
	\fxfatal{Add this after the alternative definition random walk.
	For the real references.}


	%%%%%%%%%%%%%%%%%%%%%%%%%%%%%%%%%%%%%%%%
	% Finishing the proof
	%%%%%%%%%%%%%%%%%%%%%%%%%%%%%%%%%%%%%%%%
	We can now apply Lemma \ref{L: Lemma 8} and obtain the vague convergence
	$ \Xin \rightarrow_d \Xi $.
\end{proof}


%%%%%%%%%%%%%%%%%%%%%%%%%%%%%%%%%%%%%%%%%%%%%%%%%%%%%%%%%%%%
% What's still missing
%%%%%%%%%%%%%%%%%%%%%%%%%%%%%%%%%%%%%%%%%%%%%%%%%%%%%%%%%%%%
We have now shown that our main theorem holds for all excursions
with starting times and lengths in compact intervals.
To extend this statement to the general case,
we must show that we can always find a compact subset,
such that all relevant excursions happen inside this pair of intervals.

We will show this using to approaches:
On one hand, we can analyse the behaviour of the breadth-first walk $Z_n$
after a certain stopping time.
On the other hand, 
we can consider the behaviour of components of the random graph
and the distribution of vertices on them.
\fxnote{This whole paragraph is a placeholder for something nicer.}

Consider the functions
\begin{align}
T(y) &:= \min \{s \;|\; \Wt(s)=-y\}, \label{E: def T(y)}, \\
T_n(y) &:= \min\{i \;|\; \Zn{i}=-\floor{y\n{1}{3}} \}. \label{E: def Tn(y)}.
\end{align}
Once the process $\Zn$ reaches step $T_n(y)$, 
the walk has traversed at least all vertices labeled $\{1,2,\dots,\floor{y\n{1}{3}}\}$,
since it encountered $\floor{y\n{1}{3}}$ individual components.
\fxnote{More explanation of this process. Maybe write in German then translate.}
By Theorem \ref{T: Z -> W}, $\n{-2}{3}T_n(y) \rightarrow_d T(y)$
and since $T(y) \rightarrow \infty$ as $y\rightarrow\infty$,
$T_n(y) \approx \n{2}{3}T(y) \rightarrow\infty$ as $y\rightarrow \infty$.
This is important for some reason.
\fxnote{Find out why Tn(y) is important.}


\section{Late excursions of $Z_n$}
%%%%%%%%%%%%%%%%%%%%%%%%%%%%%%%%%%%%%%%%%%%%%%%%%%%%%%%%%%%%
% SECTION: Late excursions of Z_n
%%%%%%%%%%%%%%%%%%%%%%%%%%%%%%%%%%%%%%%%%%%%%%%%%%%%%%%%%%%%

\begin{lemma}
	Let 
	$\Prob(\ExcursionEvent)$ 
	\fxnote{Is A the right letter for this?}
	be the probability that the breadth-first walk encounters an excursion $\gamma$ of length 
	$|\gamma| > \delta\n{2}{3}$, starting after step $C\n{2}{3}$.
	\fxnote{Excursions are not really encountered. Maybe traversed?}
	Then for all $\epsilon>0$ exists $C>0$ such that $\Prob(\ExcursionEvent) < \epsilon$. 
\end{lemma}

\begin{proof}
	By the law of total expectation,
	\begin{align*}
	\Prob(\ExcursionEvent) 
	&\leq \Exp{ \text{Number of excursions $\gamma$, with $|\gamma| \geq C\n{2}{3}$ and $l(\gamma) \geq \delta\n{2}{3}$ } } \\
	&= \Exp{ \sum_{\gamma: \; l(\gamma) \geq \delta\n{2}{3}} \Ind{\{|\gamma| \geq C\n{2}{3} \} } } \\
	&\leq \frac{1}{\delta^2\n{4}{3}} \Exp{ \sum_{\gamma: \; l(\gamma) \geq \delta\n{2}{3}}  |\gamma|^2 \Ind{\{|\gamma| \geq C\n{2}{3} \} }}.
	\end{align*}
	\fxnote{Fix the formatting and add equation number}
	
	Let $T$ be the time the last excursion starting before $C\n{2}{3}$ ends.
	\fxnote{Maybe diagram goes here.}
	The behaviour of the breadth-first walk after $T$ will be the same as the behaviour of a new walk on $\Gcal(n-T,p)$.
	\fxnote{Some more explanation on this, please!}
	Note that the notions of excursions of the breadth-first walk and components in the underlying graph are interchangeable.
	\fxnote{Note that the notions...}
	We write $\Ccal \in \Gcal$ to denote a component $\Ccal$ contained in the random graph $\Gcal$,
	and $|\Ccal|$ for its size.
	Then
	\begin{equation} \label{E: P(ExcEvent) 2}
	\begin{aligned}
	\Prob(\ExcursionEvent) 
	&\leq \frac{1}{\delta^2\n{4}{3}} \Exp{ \sum_{\Ccal \in \Gcal(n-T, p)}  |\Ccal|^2 \Ind{\{|\Ccal| \geq \delta\n{2}{3} \} }} \\
	&\leq \frac{1}{\delta^2\n{4}{3}} \Exp{ \sum_{\Ccal \in \Gcal(n-C\n{2}{3}, p)}  |\Ccal|^2 \Ind{\{|\Ccal| \geq \delta\n{2}{3} \} }} \\
	&\leq \frac{1}{\delta^2\n{4}{3}} \Exp{ \sum_{\Ccal \in \Gcal(n-C\n{2}{3}, p)}  |\Ccal|^2 }.
	\end{aligned}
	\end{equation}
	For ease of notation,
	let us now consider the graph $\Gcal(k,p)$.
	\begin{equation} \label{E: P(ExcEvent) 3}
	\begin{aligned}
	\Exp{\sum_{\Ccal \in \Gcal(k, p)} |\Ccal|^2 } 
	&= \Exp{ \sum_{\Ccal \in \Gcal(k, p)} |\Ccal| \sum_{v \in \Ccal} 1} \\
	&= \Exp{ \sum_{\Ccal \in \Gcal(k, p)} |\Ccal| \sum_{v \in \Gcal(k,p)} \Ind{\{v \in \Ccal\}} } \\
	&= \sum_{v \in \Gcal(k,p)} \Exp{ \underbrace{\sum_{\Ccal \in \Gcal(k, p)} \Ind{v \in \Ccal}}_{|\Ccal|} } \\ 
	&= \sum_{v \in \Gcal(k,p)} \Exp{|\Ccal(v)|} \\
	&= k \Exp{|\Ccal(1)|},
	\end{aligned}
	\end{equation}
	where $\Ccal(v)$ denotes the component containing the vertex $v$ and the last inequality stems from the interchangeability of the vertex labels.
	\fxnote{Is interchangeability of vertex labels a thing?}
	We will bound the expectation of the size of this component from above by a suitable branching process.
	Exploring the component in the language of a branching process $(Y_i, \; i\geq 0)$ works by starting at time $0$ with one vertex,
	$Y_0 = 1$.
	The number of children of this vertex is a $\Binom(k-1,p)$ distributed random variable, $Y_1$.
	In the next step, each child-vertex will itself have children,
	each Binomially distributed on the remaining set of vertices with probability $p$.
	So 
	\begin{align*}
	Y_{2,1} &\sim \Binom(k-1-Y_1,p), \\ 
	Y_{2,2} &\sim \Binom(k-1-Y_1-Y_{2,1},p), \\
	&\dots \\
	Y_{2,Y_1} &\sim \Binom(k-1-Y_1-Y_{2,1}-\dots-Y_{2,Y_1-1},p) \\
	Y_2 &= \sum_{i=1}^{Y_1} Y_{2,i}.
	\end{align*}
	The size of the component will then be the amount of explored vertices,
	or 
	To provide an upper bound,
	we consider the branching process where each number of children is $\Binom(k,p)$ distributed.
	Define the process as follows,
	\begin{equation}
	\begin{aligned}
	Z_0 &:= 1, \\
	Z_j &:= \sum_{i=1}^{Z_{j-1}} Z_{j,i},
	\end{aligned}
	\end{equation}
	 where $Z_{j,i} \sim\Binom(k,p)$ for all $i,j \geq 1$.
	 Then the process $(Y_i, \; i\geq 0)$ is stochastically dominated by 
	 $(Z_i, \; i\geq 0)$ and
	 \begin{equation}
	 |\Ccal(1)| \leq_{\text{st.}} Z_0 + Z_1 + Z_2 + \dots.
	 \end{equation}
	 For $m\geq0$,
	 \begin{align*}
	 \Exp{Z_m} 
	 &= \Exp{Z_{m-1}}kp \\
	 &\dots \\
	 &= \Exp{Z_0}(kp)^m \\
	 &= (kp)^m,
	 \end{align*}
	so
	\begin{align*}
	\Exp{|\Ccal(1)|} 
	&\leq 1 + kp + (kp)^2 + \dots \\
	&= \frac{1}{1-kp}.
	\end{align*}
	Substituting in \eqref{E: P(ExcEvent) 3} yields
	\begin{equation}
	\begin{aligned}
	\Prob(\ExcursionEvent) 
	&\leq \frac{n-C\n{2}{3}}{\delta^2 \n{4}{3}} \frac{1}{1-(n-C\n{2}{3})p} \\
	&= \frac{n-C\n{2}{3}}{\delta^2 \n{4}{3}} \frac{1}{1-(n-C\n{2}{3})(n^{-1} + t\n{-4}{3})} \\
	&= \frac{n-C\n{2}{3}}{\delta^2 \n{4}{3}} \frac{\n{1}{3}}{C-t+Ct\n{-1}{3}} \\
	&\leq \frac{n}{\delta^2 \n{4}{3}} \frac{\n{1}{3}}{C-t+Ct\n{-1}{3}} \\
	&= \frac{1}{\delta^2} \frac{1}{C-t+Ct\n{-1}{3}}.
	\end{aligned}
	\end{equation}
	For $n \rightarrow \infty$, 
	this expression converges asymptotically to $\frac{1}{\delta^2} \frac{1}{C-t}$,
	where $t$ and $\delta$ are fixed.
	Thus, for all $\epsilon>0$ we can choose $C>0$ such that
	$\Prob(\ExcursionEvent) \leq \frac{1}{\delta^2} \frac{1}{C-t+Ct\n{-1}{3}} < \epsilon$,
	which completes the proof.
\end{proof}

We have shown that with high probability there is no mass of large excursions wandering off to infinity.
\fxnote{Correct use of "with high probability"?}
\fxnote{"Mass of large excursions" needs clarification.}
For all $\epsilon>0$ we can find a $C>0$ such that,
with probability $1-\epsilon$,
all excursions of size equal to or larger than $\delta\n{2}{3}$ will start before step $C\n{2}{3}$.
For all these excursions we can apply Lemma \ref{L: Lemma 8} to prove that $\Xin \rightarrow_d \Xi$.
With these considerations we finish the proof of Theorem \ref{T: Main}.

\section{Graph-theory approach}
%%%%%%%%%%%%%%%%%%%%%%%%%%%%%%%%%%%%%%%%%%%%%%%%%%%%%%%%%%%%
% SECTION: Graph-theory approach
%%%%%%%%%%%%%%%%%%%%%%%%%%%%%%%%%%%%%%%%%%%%%%%%%%%%%%%%%%%%

%%%%%%%%%%%%%%%%%%%%%%%%%%%%%%%%%%%%%%%%%%%%%%%%%%%%%%%%%%%%
% T(y) and T_n(y)
%%%%%%%%%%%%%%%%%%%%%%%%%%%%%%%%%%%%%%%%%%%%%%%%%%%%%%%%%%%%


%%%%%%%%%%%%%%%%%%%%%%%%%%%%%%%%%%%%%%%%%%%%%%%%%%%%%%%%%%%%
% Lemma 9 - Graph Components: Statement
%%%%%%%%%%%%%%%%%%%%%%%%%%%%%%%%%%%%%%%%%%%%%%%%%%%%%%%%%%%%
\begin{lemma} \label{L: Lemma 9}
	Let $p(n, y, \delta)$ be the chance that $\Gcal \in \Gnt$
	contains a component of size greater than or equal to $\delta \n{2}{3}$
	which does not contain any vertex $i$ with $1 \leq i \leq y\n{1}{3}$.
	
	Then
	\begin{equation}
	\lim_{y \rightarrow \infty} \limsup_n p(n, y, \delta) = 0 \enspace
	\end{equation}
	for all $\delta > 0$.
\end{lemma}

%%%%%%%%%%%%%%%%%%%%%%%%%%%%%%%%%%%%%%%%%%%%%%%%%%%%%%%%%%%%
% Lemma 9 - Graph Components: Statement
%%%%%%%%%%%%%%%%%%%%%%%%%%%%%%%%%%%%%%%%%%%%%%%%%%%%%%%%%%%%
\begin{proof}
	For this proof,
	we have to change our understanding of the random graph process $\Gnt$ a little.
	Until now, we constructed a random graph by labeling all vertices $\{1, \dots, n\}$ 
	and then drawing Bernoulli random variables to construct edges and consequently components.
	Now we switch the order of actions. 
	Start with an unlabeled empty graph with $n$ vertices.
	Draw $\Bern(p)$ random variables to establish edges between vertices.
	Then, one by one, assign the labels $1$ to $n$ to the vertices,
	where the probabilities are equal for all remaining vertices.
	\fxnote{Fix this sentence.}
	Since the construction of edges is independent of vertex labels and edge are constructed independently of each other,
	these two approaches are equivalent construction of the random graph.
	\fxnote{Construction is used twice here with different meaning.}
	
	Let $\SigmaAlgebra_E$ be the sigma-algebra that includes the creation of edges and components, 
	but not the labeling of the vertices.
	\fxnote{Alter the description of the sigma-algebra.}
	For a component $\Ccal$ of size $\alpha\n{2}{3}$,
	let $v(\Ccal)$ be the label of its minimal vertex and write 
	$\ChiAn = \n{-1}{3}v(\Ccal)$.
	We show that $\ChiAn \rightarrow_d \Exponential(\alpha)$.
	\fxfatal{Proof of Chi -> Exp goes here!}
	
	Fix $\delta > 0$.
	Let $\NComp{I}$ be the expected number of components of size greater than or equal to $\delta\n{2}{3}$ 
	with its minimal vertex label in $\n{1}{3}I$.
	Then
	\begin{equation} \label{E: q(n,[y,infty])}
	\begin{aligned}
	\NComp{[y, \infty)}
	&= \ExpBig{ \ExpBig{ \sum_{\Ccal: \; |\Ccal| \geq \delta\n{2}{3}} \Ind{\{ v(\Ccal) > y\n{1}{3} \}} \cond \SigmaAlgebra_E } } \\
	&= \ExpBig{ \sum_{\Ccal: \; |\Ccal| \geq \delta\n{2}{3}} \Prob( v(\Ccal) > y\n{1}{3} \cond \SigmaAlgebra_E) } \\
	&= \ExpBig{ \sum_{\Ccal: \; |\Ccal| \geq \delta\n{2}{3}} \Prob(\ChiAn > y \cond \SigmaAlgebra_E, |\Ccal| = \alpha\n{2}{3}) }
	\end{aligned}
	\end{equation}
	\fxfatal{Fix usage of P for probability. Maybe use mathbb(P).}
	Since the distribution of $\ChiAn$ converges to the exponential distribution,
	$\Prob(\ChiAn > y) \leq e^{-\alpha y} (1+ \Smallo{1})$,
	where the last factor describes an error from the convergence.
	\fxnote{Describes an error from the convergence... Fix once convergence is proven and understood.}
	Additionally, $\Prob(\ChiAn \leq 1) \leq (1 - e^{-\alpha}) (1+ \Smallo{1})$,
	so
	\begin{equation}
	\Prob(\ChiAn > y) \leq e^{-\alpha y} (1+ \Smallo{1}) = \frac{e^{-\alpha y}}{1 - e^{-\alpha}} \Prob(\ChiAn \leq 1) (1 + \Smallo{1}).
	\end{equation}
	Using this inequality in \eqref{E: q(n,[y,infty])} leads to
	\begin{equation}
	\begin{aligned}
	\NComp{[y, \infty)}
	&\leq \ExpBig{ \sum_{\Ccal: \; |\Ccal| \geq \delta\n{2}{3}}
		\frac{e^{-|\Ccal|\n{-2}{3}y}}{1-e^{-|\Ccal|\n{-2}{3}}} \Prob(\ChiAn \leq 1) (1 + \Smallo{1}) } \\
	&\leq \sup_{\alpha \geq \delta} \frac{e^{-\alpha y}}{1-e^{-\alpha}} (1 + \Smallo{1}) 
		\ExpBig{ \sum_{\Ccal: \; |\Ccal| \geq \delta\n{2}{3}} \Prob(\ChiAn \leq 1 \cond \SigmaAlgebra_E ) } \\
	&= \frac{e^{-\delta y}}{1-e^{-\delta}} (1 + \Smallo{1}) \NComp{[0,1]},
	\end{aligned}
	\end{equation}
	Dividing by $\NComp{[0,1]}$ gives us
	\begin{equation}
	\limsup_n \frac{\NComp{[y,\infty]}}{\NComp{[0,1]}} \leq \frac{e^{-\delta y}}{1-e^{-\delta}} (1 + \Smallo{1}) \xrightarrow{y \rightarrow \infty} 0.
	\end{equation}
	For this convergence to hold,
	either $\NComp{[0,1]} \rightarrow \infty$ or $\NComp{[y,\infty]} \rightarrow 0$ as $y \rightarrow \infty$.
	Since the law of total expectation implies $p(n,y,\delta) \leq \NComp{[y,\infty]}$
	we only need to prove that
	\begin{equation}
	\sup_n \NComp{[0,1]} < \infty.
	\end{equation}
	Consulting existing literature on random graphs gives us $\sup_n \NComp{[0, \infty]} < \infty$.
	Because $\NComp{I_1} \leq \NComp{I_2}$ for $I_1 \subset I_2$,
	this concludes the proof.
	
	\fxfatal{Add reference to literature and proof of claim.}
\end{proof}








\section{$l^2$-approach}
%%%%%%%%%%%%%%%%%%%%%%%%%%%%%%%%%%%%%%%%%%%%%%%%%%%%%%%%%%%%
% SECTION: l2-approach
%%%%%%%%%%%%%%%%%%%%%%%%%%%%%%%%%%%%%%%%%%%%%%%%%%%%%%%%%%%%
\fxnote{This section will probably be deleted.}