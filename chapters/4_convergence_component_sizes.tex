% Chapter 4: Convergence of component sizes
% Contains:
%   Explanation on what's still missing
%   The deterministic lemma (Lemma 7)
%   Applying Lemma 7 (Lemma 8)
%   Graph theory approach:
%     Lemma 9
%   l2 approach:
%     Definitions on l2
%     Deterministic lemma (Lemma 14)
%     Proposition 15
%     Applying Prop 15 to the bf-walk

\chapter{Convergence of component sizes}
\todo{Update title.}

%%%%%%%%%%%%%%%%%%%%%%%%%%%%%%%%%%%%%%%%%%%%%%%%%%%%%%%%%%%%
% What's still missing
%%%%%%%%%%%%%%%%%%%%%%%%%%%%%%%%%%%%%%%%%%%%%%%%%%%%%%%%%%%%


%%%%%%%%%%%%%%%%%%%%%%%%%%%%%%%%%%%%%%%%%%%%%%%%%%%%%%%%%%%%
% Deterministic Lemma 7
%%%%%%%%%%%%%%%%%%%%%%%%%%%%%%%%%%%%%%%%%%%%%%%%%%%%%%%%%%%%
\begin{lemma} \label{L: Deterministic Lemma}
	Let $f:[0, \infty) \rightarrow \Real$ be a continuous function. 
	Let $\mathcal{E}$ be the set of non-empty intervals 
	$e=(l,r) \subset \Real_{\geq 0}$
	such that
	\begin{equation} \label{E: f cond 1}
	f(r) = f(l) = \min_{s \leq l} f(s),
	\end{equation}
	\begin{equation} \label{E: f cond 2}
	f(l) < f(s) \quad \forall l < s < r.
	\end{equation}
	Define $\Xi := \lbrace (l, r-l) \; | \; (l, r) \in \mathcal{E} \rbrace$.
	
	Suppose that for intervals $e_1, e_2 \in \mathcal{E}$ we have 
	\begin{equation} \label{E: f cond f(l1) > f(l2)}
	f(l_1) > f(l_2)
	\end{equation}
	and the complement of $\cup_{e \in \mathcal{E}} (l,r)$ has Lebesgue measure zero,
	\begin{equation} \label{E: f cond complement zero}
	\mu \left( \left( \cup_{e \in \mathcal{E}} (l,r) \right)^c\right) = 0.
	\end{equation}
	
	Let $f_n \rightarrow f$ as $n \rightarrow \infty$ uniformly on bounded intervals.
	Suppose $(\tn{i}, i \geq 1)$ satisfy the following conditions:
	
	\begin{equation} \label{E: fn cond 3}
	0 = \tn{1} < \tn{2} < ... \; \text{and} \; \lim_{i \rightarrow \infty} \tn{i} = \infty,
	\end{equation}
	\begin{equation} \label{E: fn cond 4}
	f_n(\tn{i}) = \min_{u \leq \tn{i}}f_n(u), 
	\end{equation}
	\begin{equation} \label{E: fn cond 5}
	\max_{i: \tn{i} \leq s_0}(f_n(\tn{i}) - f_n(\tn{i+1})) \rightarrow 0 \; \text{as} \; n \rightarrow \infty, \; \text{for all} \; s_0 \leq \infty.
	\end{equation}
	
	Define $\Xin := \lbrace (\tn{i}, \tn{i+1} - \tn{i}) \; | \; i \geq 1 \rbrace$.
	Then $\Xin \rightarrow \Xi$ as $n \rightarrow \infty$.
\end{lemma}
\todo{Conditions in Lemma -> Ask Sapozhnikov}
\begin{proof}
	This is the proof of Lemma \ref{L: Deterministic Lemma}. 
\end{proof}


%%%%%%%%%%%%%%%%%%%%%%%%%%%%%%%%%%%%%%%%%%%%%%%%%%%%%%%%%%%%
% Lemma 8: Applying Lemma 7 to Wt and Z_n
%%%%%%%%%%%%%%%%%%%%%%%%%%%%%%%%%%%%%%%%%%%%%%%%%%%%%%%%%%%%
The following Lemma will now link $Z_n$ and $\Wt$ in the language of Lemma \ref{L: Deterministic Lemma}
Define $\gamma (n, i) \in \{1, \dots, n\}$ to be the index that makes
$v(\gamma(n,i))$ is the last vertex of the $i-1$-st component encountered by the breadth-first walk $Z_n$.
\todo{Text clunky here.}
Let $\Ci{n,i}$ be the size of the $i$-th component.
\todo{Fix i indices}

\begin{lemma} \label{L: Lemma 8}
	Let $\Xi$ be the point process with points corresponding to excursions of $\Bt$,
	\begin{equation}
	\Xi := \{ (l(\gamma), |\gamma|) \; | \; \gamma \; \text{excursion of} \; \Bt \}.
	\end{equation}
	Let $\Xin$ be the point process with points corresponding to excursions of the breadth-first walk
	\begin{equation}
	\Xin := \{ ( \n{-2}{3} \gamma(n,i), \n{-2}{3} \Ci{n,i} ) \; | \; i \geq 1 \}.
	\end{equation}
	Then $\Xin \rightarrow \Xi$ as $n \rightarrow \infty$.
\end{lemma}
\todo{Try not to repeat the same sentence twice here.}
\begin{proof}
\end{proof}

















\section{Graph-theory approach}

\section{$l^2$-approach}