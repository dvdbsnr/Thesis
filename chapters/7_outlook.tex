% Chapter 5: Outlook
% Contains:
%   The non-uniform case
%   The multiplicative coalescence
%   Open questions?

\chapter{Outlook} \label{C: outlook}

This chapter will provide an overview over the other central issue of this paper,
which is the multiplicative coalescence.

\section{The multiplicative coalescence}
%%%%%%%%%%%%%%%%%%%%%%%%%%%%%%%%%%%%%%%%%%%%%%%%%%%%%%%%%%%%%%%%%%%%%%%%%%%%%%%%
% SECTION: The multiplicative coalescence
%%%%%%%%%%%%%%%%%%%%%%%%%%%%%%%%%%%%%%%%%%%%%%%%%%%%%%%%%%%%%%%%%%%%%%%%%%%%%%%%

So far, we always considered $\pp$ as a constant which lead to a change of $\p$ only in $n$.
We will now introduce a process that describes the random graph $\Wrg(\x, \q)$ for fix $\x$ and therefore $n$
but variable $\q$, and therefore $p_{i,j}$.

We change the notation slightly.
Fix $\x \in \ld$.
For a pair $(i,j)$, with $i < j$, 
define an exponentially distributed random variable $\xi_{i,j} \sim \Exponential(1)$,
independent for distinct pairs.
For $\pp \in [0, \infty)$, define the random graph model $\Wrg(\x, \pp)$ as the graph where there exists an edge $(i,j)$ 
iff $\xi_{i,j} \leq \pp x_i x_j$.




\section{Further results}
%%%%%%%%%%%%%%%%%%%%%%%%%%%%%%%%%%%%%%%%%%%%%%%%%%%%%%%%%%%%%%%%%%%%%%%%%%%%%%%%
% SECTION: Further results
%%%%%%%%%%%%%%%%%%%%%%%%%%%%%%%%%%%%%%%%%%%%%%%%%%%%%%%%%%%%%%%%%%%%%%%%%%%%%%%%