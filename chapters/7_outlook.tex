% Chapter 5: Outlook
% Contains:
%   The non-uniform case
%   The multiplicative coalescence
%   Open questions?

\chapter{Outlook} \label{C: outlook}

This chapter will provide an overview over the other central issue of this paper,
which is the multiplicative coalescence.

\section{The multiplicative coalescence}
%%%%%%%%%%%%%%%%%%%%%%%%%%%%%%%%%%%%%%%%%%%%%%%%%%%%%%%%%%%%%%%%%%%%%%%%%%%%%%%%
% SECTION: The multiplicative coalescence
%%%%%%%%%%%%%%%%%%%%%%%%%%%%%%%%%%%%%%%%%%%%%%%%%%%%%%%%%%%%%%%%%%%%%%%%%%%%%%%%

So far, we always considered $\pp$ as a constant which lead to a change of $\p$ only in $n$.
We will now introduce a process that describes the random graph $\Wrg(\x, \q)$ for fix $\x$ and therefore $n$
but variable $\q$, and therefore $p_{i,j}$.

We change the notation slightly.
Fix $\x \in \ld$.
For a pair $(i,j)$, with $i < j$, 
define an exponentially distributed random variable $\xi_{i,j} \sim \Exponential(1)$,
independent for distinct pairs.
For $\pp \in [0, \infty)$, define the random graph model $\Wrg(\x, \pp)$ as the graph where there exists an edge $(i,j)$ 
iff $\xi_{i,j} \leq \pp x_i x_j$.
This way, we constructed a random graph process $(\Wrg(\x, \pp); 0 \leq \pp < \infty)$ 
where the number of edges of $\Wrg(\x, \pp)$ is increasing in $\pp$.
Let $X_i(\x, \pp)$ be the size of the $i$-th largest component of $\Wrg(\x, \pp)$ and let
\begin{equation}
	\X(\x, \pp) = (X_i(\x, \pp); i \geq 1).
\end{equation}
\fxnote{This last sentence is 1:1 paper page 818.}

For a vector of vertex sizes $\x$, $(\X(\x, \pp); 0 \leq \pp < \infty)$ is a continuous time Markov chain
on a finite state space which is entirely dependent on $\x$.
A state at some time $\pp_0$ is a vector of component sizes, which consist of the sizes of the individual vertices:
\begin{equation}
	\X(\x, \pp_0) = (X_1(\x, \pp_0), X_2(\x, \pp_0), \dots) = (|\Ccal_1|, |\Ccal_2|, \dots)
\end{equation}
where $|\Ccal_i| = \sum_{v \in \Ccal_i} x_v$.

The dynamics of this Markov chain can be described as follows.
With increasing $\pp$ we expect new edges to form and therefore separate components to merge into larger components.
Take two components of sizes $|\Ccal_1| = x$ and $|\Ccal_2| = y$.
There are $xy$ different possible edges between these two that may form to directly connect them 
and create a large component of size $x+y$.
There is also the possibility that both form connections to a third component $\Ccal_3$, 
which this way merges the three into one cluster.

\begin{figure}[H]
	\centering
	\begin{tikzpicture}[level distance = 11mm, scale = 1]
	\tikzstyle{level 1}=[sibling distance=8mm]
	\tikzstyle{level 2}=[sibling distance=12mm]
	\tikzstyle{level 3}=[sibling distance=10mm]
	
	\node [plain] (1) {$v_1$} [grow=up]
	child { node [plain] (2) {$v_2$}
		child { node [plain] (3) {$v_3$}
		}
	}
	;
	\node [plain] [right=1.5cm of 1, yshift=0.55cm] (4) {$v_4$} [grow=up]
	child { node [plain] (5) {$v_5$} }
	;
	\draw [dotted] (1) -- (4);
	\draw [dotted] (1) -- (5);
	\draw [dotted] (2) -- (4);
	\draw [dotted] (2) -- (5);
	\draw [dotted] (3) -- (4);
	\draw [dotted] (3) -- (5);
\end{tikzpicture}
	\caption{Two components of size $x$ and $y$ have $xy$ possible edges between them.}
	\label{F: components}
\end{figure} 

\section{Further results}
%%%%%%%%%%%%%%%%%%%%%%%%%%%%%%%%%%%%%%%%%%%%%%%%%%%%%%%%%%%%%%%%%%%%%%%%%%%%%%%%
% SECTION: Further results
%%%%%%%%%%%%%%%%%%%%%%%%%%%%%%%%%%%%%%%%%%%%%%%%%%%%%%%%%%%%%%%%%%%%%%%%%%%%%%%%