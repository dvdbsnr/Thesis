% Chapter 7: Outlook
% Contains:
%   The multiplicative coalescent
%   Open questions?

\chapter{Outlook} \label{C: outlook}

This chapter will provide an overview over the other central issue of \cite{Aldous.1997},
the multiplicative coalescent introduced in Chapters 1.5 and 4.

So far, we always considered $\pp$ as a constant which lead to a change of $\p$ only in $n$.
We will now introduce a process that describes the random graph $\Wrg(\x, \q)$ for fixed $\x$ and $n$
but variable $\q$ and therefore $p_{i,j}$.

We change the notation slightly.
Fix $\x \in \ld$.
For a pair $(i,j)$, with $i < j$, 
define an exponentially distributed random variable $\xi_{i,j} \sim \Exponential(1)$,
independent for distinct pairs.
For $\pp \in [0, \infty)$, define the random graph model $\Wrg(\x, \pp)$ as the graph where there exists an edge $(i,j)$ 
iff $\xi_{i,j} \leq \pp x_i x_j$.
This way, we constructed a random graph process $(\Wrg(\x, \pp); 0 \leq \pp < \infty)$ 
where the number of edges of $\Wrg(\x, \pp)$ is increasing in $\pp$.
Let $X_i(\x, \pp)$ be the size of the $i$-th largest component of $\Wrg(\x, \pp)$ and let
\begin{math}
	\X(\x, \pp) = (X_i(\x, \pp); i \geq 1).
\end{math}
\fxnote{This last sentence is 1:1 paper page 818.}

For a vector of vertex sizes $\x$, $(\X(\x, \pp); 0 \leq \pp < \infty)$ is a continuous time Markov chain
on a finite state space which is entirely dependent on $\x$.
A state at some time $\pp_0$ is a vector of component sizes, which consist of the sizes of the individual vertices:
\begin{equation}
	\X(\x, \pp_0) = (X_1(\x, \pp_0), X_2(\x, \pp_0), \dots) = (|\Ccal_1|, |\Ccal_2|, \dots)
\end{equation}
where $|\Ccal_i| = \sum_{v \in \Ccal_i} x_v$.

The dynamics of this Markov chain can be described as follows.
With increasing $\pp$ we expect new edges to form and therefore separate components to merge into larger components.
Take two components of sizes $|\Ccal_1| = x$ and $|\Ccal_2| = y$.
There are $xy$ different possible edges between these two that may form to directly connect them 
and create a large component of size $x+y$.
We call a Markov process with finite state space in $\ld$ and dynamics as described above a \emph{multiplicative coalescent}.
For an initial state $\x = (x, 0, 0, \dots)$ we have a "constant" multiplicative coalescent with
\begin{equation}
	\X(\pp) = (x, 0, 0, \dots) \enspace \text{for all $-\infty < \pp < \infty$}.
\end{equation}

We say a vector $\x \in \ld$ is of finite length if $x_i = 0$ for all $i \geq N$ for some $N \in \Nat$.
For a size vector $\x$ of finite length,
the dynamics of the multiplicative coalescent starting in $\X(0) = \x$ can be expressed in martingale form as follows.
For any vector $\x$, let $\x^{(i+j)}$ be the configuration obtained by merging the $i$-th and $j$-th components of $\x$,
such that there is a new cluster of size $x_i + x_j$ inserted next to some cluster $x_u$:
\begin{equation}
	\x^{(i+j)} = (x_1, \dots, x_{u-1}, x_i + x_j, x_u, \dots, x_{i-1}, x_{i+1}, \dots, x_{j-1}, x_{j+1}, \dots).
\end{equation}
Define the filtration $\F{\pp} = \sigma\{ \X(u); u \leq \pp \}$.
Then for all test functions $g: \ld \rightarrow \Real$ we have
\begin{equation}
	\Exp{\Delta g(\X(\pp)) \cond \F{\pp}}
	= \sum_i \sum_{j > i} X_i(\pp) X_j(\pp) \left( g(\X^{(i+j)}(\pp) - g(\X(\pp)) \right)d\pp,
\end{equation}
where we use the infinitesimal notation
\begin{equation}
\begin{aligned}
	&\qquad \quad \Exp{\Delta Y(t) \cond \F{t}} = A(t)dt \\ 
	&\iff M(t) = Y(t) - \int_{0}^{t}A(s)ds \enspace \text{is a local martingale.}
\end{aligned}
\end{equation}

A key result of \cite{Aldous.1997} is the existence of at least one multiplicative coalescent process, 
the steps of which are distributed as the limit process found in Theorem~\ref{T: Main}.
We call a multiplicative coalescent \emph{eternal} if it is defined for all $-\infty < \pp < \infty$.

\begin{theorem}
	There exists a multiplicative coalescent process
	$ (\X^*(\pp); -\infty < \pp < \infty)$ on $\ld$,
	called the \emph{standard eternal multiplicative coalescent},
	such that for each $\pp$ we have $\X^*(\pp) =_d \Ctbold$,
	where $\Ctbold$ is the joint distribution of excursion lengths of $\Bt$ as used in Theorem~\ref{T: Main}.
\end{theorem}

A proof of this theorem is featured in Chapter 4 of \cite{Aldous.1998}.
A further analysis of multiplicative coalescent processes can be found in a companion paper by David Aldous and Vlada Limic, \cite{Aldous.1998}.





