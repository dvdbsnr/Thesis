% Chapter 2: Preliminaries
% Contains:
%   Weak convergence
%   Vague convergence
%   etc.

\chapter{Preliminaries} \label{C: preliminaries}
\fxnote{Update title.} 

\section{Weak convergence of probability measures}
%%%%%%%%%%%%%%%%%%%%%%%%%%%%%%%%%%%%%%%%%%%%%%%%%%%%%%%%%%%%
% SECTION: On weak convergence
%%%%%%%%%%%%%%%%%%%%%%%%%%%%%%%%%%%%%%%%%%%%%%%%%%%%%%%%%%%%

The central notion of this thesis is the convergence in distribution of random variables in general metric spaces.

\begin{definition}[Weak convergence, {\cite[p.7]{Billingsley.1999}}]
	Let $\mu_n$, $\mu$ be measures on a metric space $S$ with associated \Bosi $\mathcal{S} = \Bo(S)$.
	
	Denote by $\Cb(S)$ the space of bounded, continuous functions $f: S \rightarrow \Real$.
	
	We say $\mu_n$ converges weakly to $\mu$, $\mu_n \Rightarrow \mu$, if
	\begin{equation}
		\int_S fd\mu_n \rightarrow \int_s fd\mu
	\end{equation}
	as $n \rightarrow \infty$ for all $f \in \Cb(S)$.
\end{definition}

For a random variable $X$ in $(S, \mathcal{S})$, 
defined on a probability space $\ProbSpace$,
the distribution or law $P$ of $X$ is defined by
\begin{equation}
	P(A) := \Prob(X \in A)
\end{equation}
for all $A \in \mathcal{S}$.

For a series of random variables in the same metric space to converge in distribution 
does not require them to be defined on the same probability space.
Let $X_n$, $X$ be random variables in a metric space $(S, \mathcal{S})$,
\begin{equation}
\begin{aligned}
	X: \ProbSpace &\rightarrow (S, \mathcal{S}), \\
	X_i: (\Omega_i, \SigmaAlgebra_i, \Prob_i) &\rightarrow (S, \mathcal{S}),
\end{aligned}
\end{equation}
with associated distributions $P_n$, $P$.

Now $X_n$ converges in distribution to $X$, $X_n \rightarrow_d X$,
if their distribution functions converge weakly, $P_n \Rightarrow P$.






\begin{definition}[Convergence in distribution]
	
	
\end{definition}



\begin{definition}[Weak convergence]
	Let $S$ be a metric space and $\mathcal{S}$ it's class of Borel sets.
	Let $P, P_1, P_2, \dots$ be probability measures on defined on $S$.
	We say $P_n$ converges weakly to $P$, $P_n \Rightarrow P$, if
	\begin{equation} 
	\int_S fdP_n \xrightarrow{n \rightarrow \infty} \int_S fdP
	\end{equation}
	holds for all continuous, bounded functions $f: S \rightarrow \Real$.
\end{definition}

\begin{definition}[Convergence in distribution] \label{D: Convergence in Distribution}
	
\end{definition}

\begin{definition}[Relative compactness] \label{D: Rel Compactness}
	Inhalt...
\end{definition}

\begin{definition}[Tightness] \label{D: Tightness}
	Inhalt...
\end{definition}

\begin{theorem}[Prohorov's theorem] \label{T: Prohorov}
	Let $P_n$ be a series of probability measures on some metric space $S$. If $P_n$ is tight, then it is relatively compact.
\end{theorem}

\begin{lemma}
	In function spaces, tightness and convergence in finite dimensional distributions implies convergence in distribution.
\end{lemma}



\section{The space $\DT$}
%%%%%%%%%%%%%%%%%%%%%%%%%%%%%%%%%%%%%%%%%%%%%%%%%%%%%%%%%%%%
% SECTION: The space D
%%%%%%%%%%%%%%%%%%%%%%%%%%%%%%%%%%%%%%%%%%%%%%%%%%%%%%%%%%%%

\begin{definition}[The space $\DT$]
	
\end{definition}

Compact sets of $\DT$.

\begin{definition}[Skorohod metric]
	
\end{definition}

\begin{lemma}[Tightness in $\DT$]
	
\end{lemma}



\section{Point processes}
%%%%%%%%%%%%%%%%%%%%%%%%%%%%%%%%%%%%%%%%%%%%%%%%%%%%%%%%%%%%
% SECTION: Point processes
%%%%%%%%%%%%%%%%%%%%%%%%%%%%%%%%%%%%%%%%%%%%%%%%%%%%%%%%%%%%

\begin{definition}[Point process, {\cite[p.123]{Resnick.2008}}]
	Let $S$ be a locally compact second countable Hausdorff space
	and $\Bo(S)$ its \Bosi.
	Let $\{x_i, i \geq 1\}$ be a collection of points of $S$.
	Let
	\begin{equation}
		\mu := \sum_{i \geq 1} \delta_{x_i},
	\end{equation}
	with $\delta_x$ the Dirac measure of $x \in S$,
	be locally compact, that is, 
	if $C \in \Bo(S)$ is compact then $\mu(C) < \infty$.
	Then $\mu$ is a \emph{point measure} on $E$.
	
	Denote by $M_p(S)$ the space of all point measures on $E$
	and let $\mathscr{M}_p(S)$ be the smallest $\sigma$-algebra containing all sets of the form
	\begin{equation*}
		\{ m \in M_p(S) \cond m(A) \in B \}
	\end{equation*}
	for some $A \in \Bo(S)$ and $B \in \Bo(\Rplus)$.
	
	A \emph{point process} $N$ is a measurable map from a probability space
	$\ProbSpace$ into $(M_p(S), \mathscr{M}_p(S))$.
\end{definition}

For all coming observations, we will take $\Rplus$ as th underlying Hausdorff space with its \Bosi $\Bo = \Bo(\Rplus)$.

\begin{definition}[Poisson point process, {\cite[p.130]{Resnick.2008}}]
	We call a point process $N$ a \emph{Poisson point process} or \emph{Poisson process},
	if
	\begin{enumerate}
		\item for all disjoint sets $A_1, A_2, \dots, A_n \in \Bo$
			the random variables $N(A_1), N(A_2), \dots, N(A_n)$ are independent and
		\item for all $A \in \Bo$, $N(A)$ has Poisson distribution $\Poisson(\gamma)$,
			\begin{equation*}
				\Prob(N(A) = k) = \frac{\gamma^k}{k!}\exp(-\gamma)
			\end{equation*}
			where $\gamma = \gamma(A) \in [0, \infty]$ is the \emph{mean measure} or \emph{intensity} of $N$.
	\end{enumerate}
	The mean measure is often given in terms of a \emph{rate} or \emph{conditional intenstiy} $\lambda$ by
	\begin{equation}
		\gamma(A) = \int_A \lambda(t)dt.
	\end{equation}	
\end{definition}

\begin{definition}[Simple point process, {\cite[p.124]{Resnick.2008}}]
	A point process $N$ on $\Rplus$ is called \emph{simple} if 
	\begin{equation}
		\Prob( N({x}) > 1 ) = 0
	\end{equation}
	for all $x \in \Rplus$.
\end{definition}

By \cite[Remark 2.1, p.34]{Haenggi.2013} a Poisson point process is simple if and only if
its mean measure $\gamma$ has no discrete component,
that is $\gamma(x) = 0$ for all $x \in \Rplus$.


\begin{definition}[Vague convergence]
	Let $\CK(\Rplus)$ be the space of continuous real valued functions on $\Rplus$ with compact support,
	meaning there exists a compact set $K \in \Bo$ such that $f(x) = 0$ for all $x \notin K$.
	
	Let $\mu, \mu_1, \mu_2, \dots$ be point measures on a Hausdorff space $E$.
	We say $\mu_n$ converges vaguely to $\mu$, $\mu_n \rightarrow_v \mu$, if
	\begin{equation}
		\int_{\Rplus} fd\mu_n \xrightarrow{n \rightarrow \infty} \int_{\Rplus} fd\mu
	\end{equation}
	for all $f \in \CK(\Rplus)$.
\end{definition}


\begin{lemma}[Equivalent conditions for vague convergence, {\cite[Proposition 3.12, p.142]{Resnick.2008}}]
	Let $\mu, \mu_1, \mu_2, \dots$ be point measures on $E$.
	The following are equivalent:
	\begin{enumerate}
		\item $\mu_n \rightarrow_v \mu$,
		\item $\mu_n(B) \rightarrow \mu(B)$ for all relatively compact (i.e. with compact closure) $B$ 
			for which $\mu(\partial(B)) = 0$.
		\item
	\end{enumerate}
\end{lemma}

\begin{lemma}[Pointwise convergence, {\cite[Proposition 3.13, p.144]{Resnick.2008}}]
	Let $\mu, \mu_1, \mu_2, \dots$ be point measures on $E$ and $\mu_n \rightarrow_v \mu$.
	For compact $K$ with $\mu(\partial K) = 0$ and $n \geq N(K)$
	there exist a labeling of points of $\mu_n$ and $\mu$ in $K$ such that
	\begin{equation}
	\begin{aligned}
		\mu_n(\cdot \cap K) &= \sum_{i=1}^{M} \delta_{x_i^{(n)}}, \\
		\mu(\cdot \cap K) &= \sum_{i=1}^{M} \delta_{x_i},
	\end{aligned}
	\end{equation}
	and in $E^M$
	\begin{equation}
		( x_i^{(n)}, 1 \leq i \leq M ) \xrightarrow{n \rightarrow \infty}
		( x_i, 1 \leq i \leq M  ) 
	\end{equation}
	in the sense of componentwise convergence.
\end{lemma}


\section{Further results}
%%%%%%%%%%%%%%%%%%%%%%%%%%%%%%%%%%%%%%%%%%%%%%%%%%%%%%%%%%%%
% SECTION: Further results
%%%%%%%%%%%%%%%%%%%%%%%%%%%%%%%%%%%%%%%%%%%%%%%%%%%%%%%%%%%%

We state the theorem here, without proof, as it appears in \cite[Theorem 1.4, p.339 f.]{Ethier.2005},
omitting one of two equivalent conditions and all references to higher dimensional processes,
in order to focus on the one-dimensional case we will need for our proof.

%%%%%%%%%%%%%%%%%%%%%%%%%%%%%%%%%%%%%%%%%%%%%%%%%%%%%%%%%%%%
% Central limit theorem for martingales: Statement
%%%%%%%%%%%%%%%%%%%%%%%%%%%%%%%%%%%%%%%%%%%%%%%%%%%%%%%%%%%%
\begin{theorem}[Central limit theorem for martingales] \label{T: functional CLT martingales}
	Let $\{\Fn{t}\}$ be a filtration and $M_n$ a $\{\Fn{t}\}$-local martingale with sample paths in $D_{\Real}[0,\infty)$ and $\Mn{0}=0$.
	Let $B_n$ be a process with sample paths in $D_{\Real}[0,\infty)$, increasing in $t$, such that $M_n^2 - B_n$ is an $\{\Fn{t}\}$-local martingale.
	
	Let the following conditions hold:
	For each $T>0$,
	\begin{equation} \label{E: cond1 CLT}
	\lim_{n->\infty} \ExpBig{
		\sup_{t \leq T} | \Bn{t} - \Bn{t-}|
	} = 0,
	\end{equation}
	\begin{equation} \label{E: cond2 CLT}
	\lim_{n->\infty} \ExpBig{
		\sup_{t \leq T} | \Mn{t} - \Mn{t-}|^2
	} = 0,
	\end{equation}
	and with $c(t)$ a continuous, increasing function on $[0, \infty)$, $c(0) = 0$, let
	\begin{equation} \label{E: cond3 CLT}
	\Bn{t} \longrightarrow_p c(t).
	\end{equation}
	Then $M_n \longrightarrow_d X$ where $X$ is a process with sample paths in $C_{\Real}[0,\infty)$ and independent Gaussian increments.
\end{theorem}

We conclude this chapter by introducing Girsanovs Theorem,
which will enable us to prove certain properties of the Brownian motion with drift.
We state the theorem as it appears in \cite[Theorem 4.2.2, p.66]{Lamberton.2000}.

\begin{theorem}[Girsanov] \label{T: Girsanov}
	Let $(W(s))_{0 \leq s \leq T}$ be a Brownian motion on a probability space $(\Omega, \mathcal{F}, \mathbb{P})$.
	Let $(\Theta(s))_{0 \leq s \leq T}$ be an adapted process satisfying
	\begin{equation} \label{E: Girsanov cond Theta}
	\int_{0}^{T} \Theta^2(u)du < \infty.
	\end{equation}
	Define
	\begin{align}
	\tilde{W}(s) &:= W(s) + \int_0^s \Theta(u)du, \label{E: Girsanov def W tilde} \\ 
	X(s) &:= \exp \left\{ -\int_{0}^{s} \Theta(u) dW(u) - \frac{1}{2} \int_0^s \Theta^2(u)du \right\}. \label{E: Girsanov def X}
	\end{align}
	If $X(s)$ a martingale, that is $\Exp{X(s)} = \Exp{X(0)} = 1$ for all $s$,
	the measure $\mathbb{Q}$, defined by
	\begin{equation} \label{E: Girsanov def P tilde}
	\tilde{\mathbb{P}}(A) := \int_A X(\omega) d\mathbb{P}(\omega), \; \text{for all} \; A \in \mathcal{F},
	\end{equation}
	is a probability measure under which the process 
	$(\tilde{W}(s))_{0 \leq s \leq T}$
	is a Brownian motion.
\end{theorem}

By \cite[Remark 4.2.3, p.66]{Lamberton.2000}, a sufficient condition for $X(t)$ to be a martingale is the so-called Novikov-condition
\begin{equation} \label{E: Novikov}
\ExpBig{ \exp \left( \frac{1}{2} \int_{0}^{T} \Theta^2(u)du \right) } < \infty.
\end{equation}

We can apply this Theorem to the Brownian motion with drift $W^t$ as follows:
Recall the definition
\begin{equation}
\Wt(s) = W(s) + ts - \frac{1}{2} s^2 = W(s) + \int_0^s (t-u)du.
\end{equation}
For $T<\infty$, $\Theta(u) := t-u$ satisfies \eqref{E: Novikov},
therefore $X(t)$, as defined in \eqref{E: Girsanov def X}, is a martingale and
$\Wt$ is a standard Brownian motion under the probability measure $\tilde{\mathbb{P}}$ defined in \eqref{E: Girsanov def P tilde}.
Since $X(s) > 0$ almost surely for all $s$, the probability measures $\mathbb{P}$ and $\tilde{\mathbb{P}}$ agree which sets have probability zero:
\begin{equation} \label{E: P0 = Q0}
\mathbb{P}(A) = 0 \iff \tilde{\mathbb{P}}(A) = 0, \; \text{for all} \; A \in \mathcal{F}.
\end{equation}
Properties holding with probability $1$ or $0$ for a standard Brownian motion will hold with probability $1$ or $0$, respectively,
for $\Wt$ under $\tilde{\mathbb{P}}$ and hence under the original measure $\mathbb{P}$.
