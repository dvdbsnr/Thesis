% Chapter 3: Surplus edges
% Contains:
%   Definition of surplus edges
%   Eligible edges for s.e.
%   Calculating the probability of s.e.
%   The proof of (Z_n, N_n) => (W,N)
%   Why the overestimated probability is OK

\chapter{Surplus edges}
\fxnote{Update title.}

\section{Counting surplus edges}
\fxnote{Update title.}

\begin{figure}
	\begin{tikzpicture}[level distance = 25mm, scale = 1]
% GRAPH
\tikzstyle{level 1}=[sibling distance=22mm]
\tikzstyle{level 2}=[sibling distance=9mm]
\tikzstyle{level 3}=[sibling distance=4mm]
\node (A) [explored] {$v_1$} [grow=right]
	child { node [discovered] {$v_4$}
		child { node [neutral] {}
			edge from parent [dotted]}
		child { node [neutral] {}
			edge from parent [dotted]}
	}
	child { node [discovered] {$v_3$}
		child { node [neutral] {}
			edge from parent [dotted]}
		child { node [neutral] {}
			edge from parent [dotted]}
		child { node [neutral] {}
			edge from parent [dotted]}
	}
	child { node [explored] {$v_2$}
		child { node [discovered] {$v_6$}
			child { node [neutral] {}
				edge from parent [dotted]}
			child { node [neutral] {}
				edge from parent [dotted]}}
		child { node [discovered] {$v_5$}
			child { node [neutral] {}
				edge from parent [dotted]}}
	}
;
% LEGEND NODES
\node [explored, anchor=east](Legend-E-Node) at (10,2.5){};  
\node[anchor=west](Legend-E-Text) at (10,2.5){Explored}; 
\node [discovered, anchor=east](Legend-D-Node) at (10,1.5){};  
\node[anchor=west](Legend-D-Text) at (10,1.5){Discovered}; 
\node [neutral, anchor=east](Legend-N-Node) at (10,0.5){};  
\node[anchor=west](Legend-N-Text) at (10,0.5){Neutral}; 
% LEGEND LINES
\draw (9.6,-1) -- (10.05,-1); 
\node[anchor=west](Legend-F-Text) at (10,-1){Edge found}; 
\draw [dotted] (9.6,-2) -- (10.05,-2); 
\node[anchor=west](Legend-NF-Text) at (10,-2){Edge not yet found}; 
% LEGEND BOX
\node[draw, fit=(Legend-E-Node)(Legend-N-Text)(Legend-D-Text)(Legend-NF-Text)] {};
\end{tikzpicture}
\end{figure}

\section{Weak convergence of $(Z^t_n, N^t_n)$}

\fxnote{Introductory text.}
\fxnote{Definition weak convergence}
\begin{definition}[Weak convergence]
	Let $S$ be a metric space and $\mathcal{S}$ it's class of Borel sets.
	Let $P, P_1, P_2, \dots$ be probability measures on defined on $S$.
	We say $P_n$ converges weakly to $P$, $P_n \Rightarrow P$, if
	\begin{equation} 
	\int_S fdP_n \xrightarrow{n \rightarrow \infty} \int_S fdP
	\end{equation}
	holds for all continuous, bounded functions $f: S \rightarrow \Real$.
\end{definition}

This definition now applies to convergence in distribution of random variables as follows:
Let $X, X_1, X_2, \dots$ be random variables on some probability space $S$.
Let $P, P_1, P_2, \dots$ be the corresponding distributions. 

\begin{definition}[Tightness]
	Inhalt...
\end{definition}

\begin{definition}[Relative compactness]
	Inhalt...
\end{definition}

\begin{theorem}[Prohorov's theorem]
	Let $P_n$ be a series of probability measures on some metric space $S$. If $P_n$ is tight, then it is relatively compact.
\end{theorem}



We will prove weak convergence by showing that
\begin{equation}
\Exp{ f(\bar{Z}^t_n, \bar{N}_n) } \longrightarrow \Exp{ f(W^t, N) }
\end{equation}
as $n \rightarrow \infty$, 
for all continuous, bounded functions 
$f: D^2 \rightarrow \Real$.